
A minha mãe me apoiou em minha decisão de cursar faculdade noutra cidade. E sempre que possível (mesmo com dificuldade) ajudou como pode: com móveis para meu primeiro quarto, algumas compras de mercado e algum dinheiro quando ficava mais crítico. Inclusive isso tudo pode ter sido um dos motivos da decisão dela em vender o imóvel em que morávamos.

A toda a família Lischka que me acolheu inicialmente como hóspede na residência do casal Arnaldo e Jacira (prima da minha mãe). Foi também na empresa familiar deles, juntos com seus filhos, que tive meu primeiro emprego na cidade e me foi cedido para moradia um quartinho e dependências junto à empresa.

Ao meu irmão que me empregou as sábados e minhas cunhadas que me hospedavam, além de amigos. A todos os motoristas que me deram carona, economizando assim a passagem para que eu pudesse voltar com uma grana a mais do final de semana.

A todos os meus amigos de moradia que tiveram que me aguentar nas repúblicas que habitei, especialmente à mais recentemente extinta ``Mansão Amarela''.

Ao colega acadêmico Bilecki, por ter sido para mim um exemplo de dedicação ao estudo. E que coincidentemente seguiu em sua dissertação a mesma linha que já pretendida para esta sempre que possível colaborando.

A Joseli que sua experiência passando pelo mesma situação de fim de curso foi de grande ajuda. Também seu apoio e motivação, bem como dos meu amigos de infância José e Rafael que fizeram o mesmo.

A toda a experiência social, esportiva e de conheciento que as confraternizações com outros acadêmicos, Jiudesc e os vários Latinoware proporcionaram.

As empresas Hardline e Xthor que me ofereceram estágio e emprego respectivamente. Principalmente a segunda, na qual todo aprendizado e parceria com o Richard levou ao meu crescimento profissional, acadêmico e pessoal.

Sem esquecer é claro das minhas garrafas térmicas, que mantiveram tanto café quentinho quanto suco de guaraná gelado para embalar as noites dedicadas a esta realização.
