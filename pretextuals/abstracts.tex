
% ---
% RESUMOS
% ---

% resumo em português
%MODRO: Resumo Monografia
%Qual o problema? Qual a dificuldade atual?
%Porque da solução?
%Como desenvolvido?
%Diagramas modelagem?
%Tempo de desenvolvimento?
%Processo de aprendizado?
%Ferramentas utilizadas?
%Market share?
%Funcionalidades?
%Onde testado?
%Onde utilizado?
%Casos de uso? Opiniões?
%Documentação / Projeto?
\begin{resumo}
Trata-se do estudo das opções para prática computacional no processo de ensino e aprendizagem, sugerindo um conceito de ferramenta que diferente das existentes, pois permite acesso via navegador de internet sem instalação ou download adicional. Vários conceitos são apresentados sobre algoritmos como: sua estrutura e seus tradutores, sejam eles compiladores ou interpretadores. Mas para o foco do trabalho apenas algoritmo com simples impressão em tela foi implementado, permitindo demonstrar seu diferencial que além de acesso nativo via navegador o pode ser utilizado sem conexão com a internet. Sendo o problema a atual dificuldade de iniciar a execução do programa para a prática durante o aprendizado de algoritmos, vem se propor esta solução para acesso simplifica independente da plataforma ou dispositivo do aluno. Foi desenvolvido utilizando da ferramenta utilizada nesta instituição por alguns professores como modelo, ela foi repensada modo minimalista quanto sua interface, desse pensamento o diagrama de caso de uso foi obtido. Foi necessário um processo de aprendizado sobre compiladores por não termos neste curso ainda alguma disciplina que aborde este tema. Se de modo ininterrupto pode-se concluir como tempo total de desenvolvimento três meses. Foram utilizadas técnicas de vanguarda para funcionamento em navegadores modernos, dentre essas os últimos padrões da linguagem de programação ECMAScript (popularmente JavaScript). Nessa linguagem foram encontrados componentes de código livre: Acorn e JS-Intrepreter para a análise léxico-sintática e execução semântica respectivamente, além do Ace como editor de texto.

\vspace{\onelineskip}
\textbf{Palavras-chaves}: Lógica da programação. Algoritmos. Portugol. Interpretador. HTML5.
% ensino, aprendizagem
\end{resumo}

% resumo em inglês
%\begin{resumo}[Abstract]
%\begin{otherlanguage*}{english}
%
%\vspace{\onelineskip}
%\textbf{Key-words}: Programming logic. Algorithms. Portugol. Interpreter. HTML5.
%% teaching, learning
%\end{otherlanguage*}
%\end{resumo}
