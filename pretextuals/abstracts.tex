
% ---
% RESUMOS
% ---

% resumo em português
%MODRO: Resumo Monografia
%Qual o problema? Qual a dificuldade atual?
%Porque da solução?
%Como desenvolvido?
%Diagramas modelagem?
%Tempo de desenvolvimento?
%Processo de aprendizado?
%Ferramentas utilizadas?
%Market share?
%Funcionalidades?
%Onde testado?
%Onde utilizado?
%Casos de uso? Opiniões?
%Documentação / Projeto?
\begin{resumo}
Trata-se do estudo das opções para prática computacional no processo de ensino e aprendizagem, sugerindo um conceito de ferramenta que diferente das existentes permite acesso via navegador de internet sem instalação ou download mesmo que apenas de complementos ou já disponíveis no computador. Vários conceitos são apresentados sobre algoritmos sua estrutura e seus "transformadores", sejam eles compiladores, tradutores ou interpretadores. Mas para o foco do trabalho apenas instruções simples de escrita em tela sem utilização de variáveis foi utilizada, permitindo demonstrar seu diferencial que além de acesso via navegador, nativamente (diferindo das demais soluções similares), o acesso mesmo sem conexão com a internet.

\vspace{\onelineskip}
\textbf{Palavras-chaves}: Lógica da programação. Algoritmos. Portugol. Interpretador. HTML5.
% ensino, aprendizagem
\end{resumo}

% resumo em inglês
%\begin{resumo}[Abstract]
%\begin{otherlanguage*}{english}
%
%\vspace{\onelineskip}
%\textbf{Key-words}: Programming logic. Algorithms. Portugol. Interpreter. HTML5.
%% teaching, learning
%\end{otherlanguage*}
%\end{resumo}
