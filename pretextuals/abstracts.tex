
% ---
% RESUMOS
% ---

% resumo em português
%MODRO: Resumo Monografia
%Qual o problema? Qual a dificuldade atual?
%Porque da solução?
%Como desenvolvido?
%Diagramas modelagem?
%Tempo de desenvolvimento?
%Processo de aprendizado?
%Ferramentas utilizadas?
%Market share?
%Funcionalidades?
%Onde testado?
%Onde utilizado?
%Casos de uso? Opiniões?
%Documentação / Projeto?
\begin{resumo}
%Trata-se do estudo das opções para prática computacional no processo de ensino e aprendizagem, sugerindo um conceito de ferramenta que diferente das existentes, pois permite acesso via navegador de internet sem instalação ou download adicional. Vários conceitos são apresentados sobre algoritmos como: sua estrutura e seus tradutores, sejam eles compiladores ou interpretadores. Mas para o foco do trabalho apenas algoritmo com simples impressão em tela foi implementado, permitindo demonstrar seu diferencial que além de acesso nativo via navegador o pode ser utilizado sem conexão com a internet. Sendo o problema a atual dificuldade de iniciar a execução do programa para a prática durante o aprendizado de algoritmos, vem se propor esta solução para acesso simplifica independente da plataforma ou dispositivo do aluno. Foi desenvolvido utilizando da ferramenta utilizada nesta instituição por alguns professores como modelo, ela foi repensada modo minimalista quanto sua interface, desse pensamento o diagrama de caso de uso foi obtido. Foi necessário um processo de aprendizado sobre compiladores por não termos neste curso ainda alguma disciplina que aborde este tema. Se de modo ininterrupto pode-se concluir como tempo total de desenvolvimento três meses. Foram utilizadas técnicas de vanguarda para funcionamento em navegadores modernos, dentre essas os últimos padrões da linguagem de programação ECMAScript (popularmente JavaScript). Nessa linguagem foram encontrados componentes de código livre: Acorn e JS-Intrepreter para a análise léxico-sintática e execução semântica respectivamente, além do Ace como editor de texto.
O presente trabalho trata do desenvolvimento de um protótipo para a prática computacional de algoritmos na aprendizagem de lógica da programação. O protótipo foi desenvolvido tendo como base a ferramenta já utilizada pela maioria do professores desta instituição que lecionam a disciplina de algoritmos. O principal diferencial apresentado neste protótipo é permitir o acesso via navegadores de internet nativamente, sem instalações adicionais. Outro ponto chave é a possibilidade de, após um primeiro acesso, utilizar esta aplicação mesmo quando sem conexão com a internet, devido a uma tecnologia que armazena os arquivos necessários no navegador. Sendo o problema a atual dificuldade de iniciar o programa para a prática durante o aprendizado de algoritmos, vem se propor esta solução para acesso simplificado independente da plataforma ou dispositivo do aluno. Foi mantida a mesma sintaxe já utilizada na ferramenta na qual foi este protótipo baseado, permanecendo a variação de Portugol criada em 2002. A interface gráfica foi recriada mantendo somente elementos essenciais, portanto minimalista. Neste ponto já se tem apresentado o protótipo visualmente, com o editor de código e passando para a simulação dos algoritmos. Conceitos de algoritmos foram estudados e principalmente seus tradutores, tanto compiladores quanto interpretadores. Foi necessário um processo de aprendizado sobre estes compiladores por não termos neste curso ainda alguma disciplina que aborde este tema. Foram utilizadas técnicas de vanguarda para funcionamento em navegadores modernos, dentre essas os últimos padrões da linguagem de programação ECMAScript (popularmente JavaScript). Nessa linguagem foram encontrados componentes de código livre: Acorn e JS-Interpreter para a análise léxico-sintática e execução semântica respectivamente, além do Ace como editor de texto.

\vspace{\onelineskip}
\textbf{Palavras-chaves}: Lógica da programação. Algoritmos. Portugol. Interpretador.
% ensino, aprendizagem
\end{resumo}

% resumo em inglês
%\begin{resumo}[Abstract]
%\begin{otherlanguage*}{english}
%
%\vspace{\onelineskip}
%\textbf{Key-words}: Programming logic. Algorithms. Portugol. Interpreter. HTML5.
%% teaching, learning
%\end{otherlanguage*}
%\end{resumo}
