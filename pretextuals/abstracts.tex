
% ---
% RESUMOS
% ---

% resumo em português
\setlength{\absparsep}{18pt} % ajusta o espaçamento dos parágrafos do resumo
\begin{resumo}
% TODO Reescrever
Inicialmente, este projeto inicia-se buscando demonstrar que as disciplinas
iniciais relacionadas à algoritmos, dos cursos de tecnologia da informação, que
muitas vezes são o primeiro contato do indivíduo com a programação devem unir de
maneira adequada uma aprendizagem agradável e de qualidade para manter o
interesse, melhorando o aproveitamento em conteúdo e aprimoramento. Desta
maneira, com este projeto busca-se viabilizar a implantação de facilitadores
para a Linguagem de programação denominada Portugol, da qual já é tratada no
âmbito acadêmico, mas que pode ser aprimorada em sua aplicação para o uso na
aprendizagem. Num primeiro momento, identificou-se que esta ferramenta de
simulação e interpretação de algoritmos faz-se necessária, sendo opção de cada
professor a escolha do software que prefira dentre vários disponíveis e assim,
com o objetivo de solucionar esse problema, esse trabalho propõe a criação de
um interpretador dinâmico, de fácil acesso, com a pseudolinguagem em português
entre outras funcionalidades as quais resultam no auxílio aos docentes nas
avaliações da disciplina de algoritmos em Portugol, mantendo a coerência e
padrão das avaliações. As turmas iniciais nos cursos de Sistemas de Informação e
áreas afins possuem muitas vezes mais de 40 alunos, o que pode levar o professor
a perder o padrão da correção, correndo o risco de não manter a mesma forma de
correção para todos os alunos. Esse sistema poderá ser aplicado nas turmas de
iniciação em algoritmos, e mensurado se com sua adoção as notas e aproveitamento
obtiveram acréscimo significativo e a subsequente redução do índice de
reprovações. Pensando nisto, propõe-se elaboração de um sistema que auxilia a
manter uma coerência na correção das avaliações, o qual conta com dois módulos,
sendo o Módulo do Professor e o Módulo do Aluno. O professor entra com os
enunciados dos algoritmos e as informações mais relevantes a respeito dos
mesmos. O acadêmico entra com sua resolução do problema, e o algoritmo será
avaliado utilizando compiladores, através do uso da análise léxica, sintática e
semântica, e o uso da análise pragmática, integrante da técnica de Inteligência
Artificial denominada Processamento de Linguagem Natural. Logo após será gerado
um arquivo texto com os erros encontrados, que será avaliado sendo geradas as
notas e comentários específicos. O sistema foi implementado para a linguagem de
programação Portugol, permitindo ao acadêmico realizar seus exercícios e ter uma
avaliação a respeito dos mesmos, bem como ao professor uma visão de como os
acadêmicos estão se saindo em sua disciplina.

 \textbf{Palavras-chaves}: lógica da programação, algoritmos, portugol, interpretador, html5.
 % ensino, aprendizagem
\end{resumo}

% resumo em inglês
\begin{resumo}[Abstract]
 \begin{otherlanguage*}{english}

   \vspace{\onelineskip}

   \noindent
   \textbf{Key-words}: programming logic, algorithms, portugol, interpreter, html5.
 % teaching, learning
 \end{otherlanguage*}
\end{resumo}
