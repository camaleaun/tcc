\documentclass[
	12pt,
	openright,
	oneside, % oposto a twoside
	a4paper,
	chapter=TITLE,
	section=TITLE,
	english,
	brazil % o último idioma é o principal do documento
	]{abntex2-udesc}

\usepackage{mathptmx} % Usa a fonte Times New Roman
\usepackage[T1]{fontenc} % Selecao de codigos de fonte.
\usepackage[utf8]{inputenc} % Codificacao do documento (conversão automática dos acentos)
\usepackage{indentfirst} % Indenta o primeiro parágrafo de cada seção.
\usepackage{color} % Controle das cores
\usepackage{graphicx} % Inclusão de gráficos
\usepackage{microtype} % para melhorias de justificação
\usepackage[singlelinecheck=false]{caption}

\usepackage[alf,abnt-emphasize=bf,abnt-full-initials=yes,abnt-etal-cite=2]{abntex2cite}

\setlist[itemize]{noitemsep,topsep=-12pt}

\instituicao{Universidade do Estado de Santa Catarina - UDESC}
\centro{Centro de Educação do Planalto Norte - CEPLAN}

\curso{Engenharia de Produção - Habilitação Mecânica}

\cidade{São Bento do Sul}
\uf{SC}
\local{\inserecidade, \insereuf}

\data{2016}

\titulo{Proposta de Redução de Retrabalho em uma Empresa de Chicotes Elétricos por Meio de Equipes de Alta Performance}
\autor{Joseli T. de Souza}

\orientadora{Prof.\textsuperscript{a} Fernanda Hänsch Beuren, Dra.}

\tipotrabalho{Trabalho de Conclusão de Curso}

\preambulo{Trabalho de Conclusão apresentado ao Curso de Engenharia de Produção - Habilitação Mecânica, da Universidade do Estado de Santa Catarina, como requisito para a obtenção do grau de Bacharel em Engenharia de Produção - Habilitação Mecânica}

\makeatletter
\hypersetup{
     	%pagebackref=true,
		pdftitle={\@title},
		pdfauthor={\@author},
    	pdfsubject={Redução de Retrabalho com Equipes de Alta Performance},
	    pdfcreator={LaTeX with abnTeX2},
		pdfkeywords={Equipes de alta performance,} {Retrabalho,} {Comunicação.},
		colorlinks=false,       		% false: boxed links; true: colored links
    	linkcolor=black,          	% color of internal links
    	citecolor=blue,        		% color of links to bibliography
    	filecolor=magenta,      		% color of file links
		urlcolor=blue,
		bookmarksdepth=4
}
\makeatother

\makeindex

\begin{document}
\frenchspacing

\imprimircapa

\imprimirfolhaderosto

\begin{folhadeaprovacao}

  \begin{center}
    {\bfseries\MakeTextUppercase{\imprimirautor}}

    \vspace*{\fill}\vspace*{\fill}
    \begin{center}
      \bfseries\MakeTextUppercase{\imprimirtitulo}
    \end{center}
    \vspace*{\fill}

   \end{center}

  \noindent
  \imprimirpreambulo

  \vspace*{\fill}

  \noindent
  {\bfseries Banca Examinadora}

  \vspace*{\ABNTEXsignskip}

  \setlist[description]{font=\normalfont}

  \begin{description}[labelindent=0pt,labelwidth=4cm]
    \item[Orientadora:]
    \assinaturaorientador{\textbf{Prof.\textsuperscript{a} Dra. Fernanda Hänsch Beuren \\ \insereinstituicao}}
  \end{description}

  \setlist[description]{font=\bfseries}

  \begin{description}[labelindent=0pt,labelwidth=4cm]
    \item[Membros:]
    \vspace*{\ABNTEXsignskip}
    \assinaturaorientador{\textbf{Prof. Dr. Alexandre Borges Fagundes \\ \insereinstituicao}}
    \item[]
    \vspace*{\ABNTEXsignskip}
    \assinaturaorientador{\textbf{Prof. Dr. Delcio Pereira \\ \insereinstituicao}}
  \end{description}

   \begin{center}
    \vspace*{0.5cm}
    \inserecidade, 28/11/2016
    \vspace*{1cm}
  \end{center}

\end{folhadeaprovacao}

\begin{dedicatoria}
   \vspace*{\fill}
   \hspace{.45\textwidth}
   \begin{minipage}{.5\textwidth}
      \SingleSpacing
À minha filha, pela paciência e compreensão diante da minha ausência durante todos os anos de curso.
   \end{minipage}
	\vspace*{\fill}
\end{dedicatoria}

\begin{agradecimentos}
À Deus por ter me dado saúde, força e animo para enfrentar os desafios.

À esta universidade, seu corpo docente, direção e administração que oportunizaram a construção de meu futuro profissional e crescimento pessoal.

À minha orientadora, pelo suporte, pelas suas correções e incentivos.

Aos meus pais pelo apoio.

E a todos que direta ou indiretamente fizeram parte da minha formação, muito obrigada.
\end{agradecimentos}

\begin{epigrafe}
    \vspace*{\fill}
   \hspace{.45\textwidth}
   \begin{minipage}{.5\textwidth}
      \SingleSpacing

{\small``Quando pensamos em como as pessoas trabalham, a
intuição ingênua que temos é que as pessoas são como
ratos em um labirinto - que tudo com que as pessoas se
preocupam é dinheiro, e, no momento em que damos
dinheiro às pessoas, podemos direcioná-las para o
trabalho de qualquer maneira.''

\vspace{\onelineskip}

Dan Ariely}
   \end{minipage}
	\vspace*{\fill}
\end{epigrafe}

\begin{resumo}
A presente pesquisa propõe a redução de retrabalho em um setor de uma empresa
multinacional do ramo automotivo, dedicada a produção de componentes elétricos (chicotes
elétricos) para caminhões, ônibus, tratores, colheitadeiras e carros de passeio, através da
metodologia de equipes de alta performance. Para tal, o referencial teórico desta pesquisa
compreende o estudo do trabalho em equipes de alta performance, essas equipes diferenciamse
através dos resultados alcançados. A justificativa para a formação de equipes de alta
performance inclui melhoraria na produtividade, qualidade, satisfação no trabalho e
comprometimento organizacional. A implementação destas equipes depende de métodos e
precisa ser realizada de maneira sistematizada. O objetivo deste trabalho é evidenciar as
vantagens em se trabalhar neste modelo, apresentando a metodologia aplicada para o êxito na
criação de equipes de alta performance, na gestão da equipe de trabalho no contexto geral da
empresa, tendo como principal propósito a identificação de retrabalho no processo produtivo
em uma equipe específica, buscando sua redução. Foram definidas funções, métodos de
trabalho, estabelecidos controles, criadas estratégias de solução de problemas, tornando a
empresa mais sinérgica. A partir daí a empresa pôde visualizar com maior clareza todos os
problemas que estavam afetando sua performance. Como resultado, a metodologia aplicada
permite medir e controlar os problemas operacionais, obtendo uma melhoria significativa nos
campos de qualidade, custo, prazo, comunicação e desenvolvimento pessoal.

\vspace{\onelineskip}
\textbf{Palavras-chaves}: Equipes de alta performance. Retrabalho. Comunicação.
\end{resumo}

\pdfbookmark[0]{\listfigurename}{lof}
\listoffigures*
\cleardoublepage

\pdfbookmark[0]{\listofquadrosname}{loq}
\listofquadros*
\cleardoublepage

\pdfbookmark[0]{\listtablename}{lot}
\listoftables*
\cleardoublepage

\pdfbookmark[0]{\contentsname}{toc}
\tableofcontents*
\cleardoublepage


\textual
\pagestyle{simple}

\chapter{Introdução}

O objetivo da indústria é produzir mais e melhor, sendo assim, buscam ferramentas
para alcançar este objetivo. Ao longo da história, diversas formas de otimizar o processo
produtivo já foram estudadas, e com o decorrer do tempo, a valorização do capital humano
vem aumentando. As empresas vêm buscando formas para que seus funcionários trabalhem
mais motivados, rendendo sua máxima produtividade, alcançando a excelência operacional
\cite{guelbert2009etal}.

Excelência operacional é o desenvolvimento de facilitadores para gerar benefícios
competitivos em um ambiente dinâmico com base nos recursos de uma organização. A
composição e expansão dos facilitadores é a base para a melhoria contínua, mudança e a
otimização dos processos de negócios. Portanto, é a capacidade dinâmica para realizar
processos centrais eficazes e eficientes na cadeia de criação de valor, utilizando os fatores
tecnológicos, culturais e organizacionais de forma integrada e com base na respectiva
estratégia \cite{jaeger2014etal}.

Segundo \citeonline{jaeger2014etal}, o primeiro passo para a excelência é a realização sistemática
ao invés de melhorias intuitivas. Excelência nas operações inclui um desempenho superior nas
pessoas relacionadas. As pessoas tem um papel muito importante dentro da cultura de
excelência, onde todos são capazes de inovar e poderão propor melhorias. O alinhamento das
decisões de gestão, estabelecimento da cultura, recursos de operações, competências e
capacidades está ligada através de estratégia de operações, o que constitui um elemento
central dos facilitadores de operações.

Existem três níveis para o alcance da excelência operacional. O primeiro nível:
Estratégia, a qual determina o que deve ser feito rumo as melhores práticas. O segundo nível:
Melhoria de desempenho, a qual abrange repensar vetores de desempenho e alavancas
competitivas, tais como estruturas de custos e produtividade de ativos \cite{jaeger2014etal}. As
preocupações de terceiro nível: Capacitadores é medir e controlar todos os problemas
operacionais tais como organização, processos, com valores tangíveis para a empresa
\cite{jaeger2014etal}.

Diante do exposto anteriormente, este trabalho tem o propósito de pesquisar e analisar
a busca pela excelência operacional, o impacto da aplicação de ferramentas que buscam
melhoria na eficiência produtiva, bem como, identificar práticas que levem a possíveis falhas
no processo, estudando formas de reduzir o retrabalho nas etapas da manufatura.

\section{Objetivos do Trabalho}
\subsection{Objetivo Geral}

Propor a redução de retrabalho na produção de uma empresa multinacional do ramo
automotivo, dedicada a produção de componentes elétricos (chicotes elétricos) para
caminhões, ônibus, tratores, colheitadeiras e carros de passeio, através de uma melhoria na
comunicação interna da empresa a partir da metodologia de equipes de alta performance.

\subsection{Objetivos específicos}

Na busca para alcançar o objetivo geral, os seguintes objetivos específicos são
propostos:
\begin{itemize}[noitemsep,topsep=0pt]
\item Mapear a situação atual da empresa, com foco no setor, visando identificar os
retrabalhos no processo de chicotes elétricos;
\item Aplicar a metodologia de equipes de alta performance;
\item Analisar os dados obtidos, comparando os resultados da situação atual com
aplicação da metodologia;
\item Mapear a situação proposta depois de aplicada a metodologia, apresentando
sugestões de melhorias.
\end{itemize}

\section{Justificativa do Trabalho}
Faz parte do sucesso e crescimento das organizações a busca pela excelência em suas
atividades, saber se está no caminho certo depende da avaliação de desempenho dos seus
processos. Para mensurar os problemas identificados, é necessário monitorar o que está
acontecendo no processo \cite{jaeger2014etal}.

Devido à dinâmica do ambiente em que as organizações competem, desenvolveu-se
um interesse considerável para a medição de desempenho. O principal papel da medição de
desempenho é avaliar a posição atual da organização e também ajudar os gestores a criar e
implementar uma estratégia melhor \cite{ivanov2014etal}.

Para \citeonline{katzenbach1999}, equipe e bom desempenho são duas coisas inseparáveis.
Dessa maneira, equipe de trabalho está se tornando cada vez mais importante nas
organizações modernas devido aos seus resultados e benefícios.

Em primeiro lugar, é necessário estabelecer o que se entende por desempenho da
equipe. Existem três variáveis-chave de desempenho: a qualidade dos resultados de uma
equipe, a quantidade de tempo necessário para entregar esta saída, e o custo de fazê-lo
\cite{crowder2012etal}.

Hoje em dia alcançar a excelência no desenvolvimento de produtos é considerado um
dos principais diferenciais estratégicos, ainda mais do que a capacidade de fabricação. Esta
alteração baseia-se no fato de que o desenvolvimento de novos produtos mais rapidamente do
que os concorrentes fazem é uma arma estratégica eficaz para ter sucesso em mercados cada
vez mais turbulentos \cite{tsiotras2014}.

Só quando se mede a situação atual do programa, na verdade, é possível direcionar os
esforços de melhoria \cite{tsiotras2014}. Desta forma, qualidade e abordagens de melhoria
de produtividade têm o trabalho em equipe como um dos seus pilares centrais. O
envolvimento dos trabalhadores afeta cinco principais determinantes da eficácia
organizacional: a motivação, satisfação, aceitação da mudança, resolução de problemas, e
comunicação. Sendo assim, a inclusão dos trabalhadores na definição de métodos e
procedimentos é uma forma eficaz de abordagem de melhoria, e pode motivar os funcionários
a produzirem um trabalho com melhor qualidade \cite{taveira2008}.

Outro fator importante a ser observado é que as equipes são sistemas sociais, e a
comunicação é fundamental para alto desempenho devido ao seu papel no intercâmbio de
informações e coordenar os esforços dos membros da equipe que permitam um trabalho
integrado. Devido à natureza multidisciplinar da equipe de alta performance, a comunicação
pode desempenhar um papel bastante crítico, e como tal, a comunicação deve ser entendida
como uma importante característica do modelo de equipe de alta performance. Dentro de
comunicação, um comportamento fundamental é a busca da informação \cite{crowder2012etal}.

Esta pesquisa justifica-se pelo empenho em aliar a performance da equipe à sua
eficiência, buscando criar o clima adequado para que a equipe possa trabalhar em seu
potencial máximo.

\section{Estrutura do Trabalho}

O trabalho está organizado em cinco capítulos. O Capitulo 1 apresenta a introdução
sobre o tema estudado, bem como o objetivo do trabalho destacando a importância da
pesquisa.

No Capítulo 2 foi realizada a revisão bibliográfica definindo o que são equipes de
trabalho, destacando as características das equipes de alta performance, discorrendo ainda
sobre retrabalho e a comunicação interna.

O Capítulo 3 detalha os métodos de pesquisa adotados, classificando o tipo de pesquisa
aplicado, o universo da pesquisa e as etapas seguidas.

O Capítulo 4 apresenta o estudo de caso, onde é efetivamente aplicada a pesquisa
planejada, apresentando coleta e análise de dados.

No Capítulo 5 são apresentadas as considerações finais, destacando-se pela análise dos
objetivos atingidos, descrevendo as limitações da pesquisa e sugestões para trabalhos futuros.

\postextual

\bibliographystyle{abntex2-alf}
\bibliography{references}

\end{document}
