
\chapter{Considerações Finais}

O objetivo principal deste projeto era apresentar o conceito de aplicação para prática de algoritmos sem instalação, mas permitindo \textit{offline}. Neste sentido sua estrutura como tradutor se mostrou desafiadora mesmo após a opção por componentes para montagem de editor e analisadores. Por outro lado mesmo recente a tecnologia de tratamento atual para acesso desconectado é bem acessível quanto a explicação de sua utilização, considerado como de dificuldade moderada.

Notou-se que há ruptura devido a muitas possibilidade ferramentas e variações nas suas sintaxes de pseudo-linguagem. Portugol podia ser um termo bem objetivo quando as pontos de sua estrutura e na verdade é tratado de modo irresponsável já que até hoje nenhum trabalho de especificação foi realizado.

Tentativas anteriores foram realizadas para disponibilização de acesso irrestrito via navegador de ferramenta similar. Porém o iniciante não costuma ter instalado em sua máquina \textit{Java Virtual Machine} (JVM) e mesmo os navegadores como o Google Chrome tem perdido o suporte as essas tecnologias.

Devido a análise do cenário pode-se ter além do acesso \textit{offline}, a primeira iniciativa de âmbito educacional em ensino superior, em consideração deste importante nicho dos dispositivos móveis. É gratificante ter como resultado uma aplicação que pode ser utilizada em qualquer dispositivo móvel, já que é raro quem não tenha um smartphone (principalmente entre universitários), mesmo sem se utilizar das linguagens nativas de seus sistemas operacionais.

\section{Relação entre os Objetivos e os Resultados Obtidos}

O protótipo foi construído com êxito, conforme consta definido nos objetivos. E também muito importante foi desenvolvido de maneira pública no Github e está disponível para livre utilização, constando nas referências deste trabalho.

Fazer com que sua sintaxe fosse compatível com UAL foi possível, graças a limitação pré-definida inclusive reafirmada no objetivo. Em acordo com os primeiros exercícios do ensino, se tornou como mínimo produto viável dentro da sintaxe pretendida.

Por mais que permita-se acesso via navegadores modernos de internet mesmo sem conexão com insternet ou necessidade de qualquer instalação adicional, esperava-se durante o planejamento que fosse mais ampla a compatibilidade para tais fins. Os navegadores compatíveis são afirmados como mais populares nas notícias. Porém principalmente a Microsoft deveria em breve implementar essa funcionalidade em seu navegador, o Edge em substituição ao IE, pois é considerado promissor nesse sentido e pré instalado por padrão em seu popular sistema operacional Windows na versão 10.

\section{Limitações da Pesquisa}

% NOTE Dificuldade em identificar material
Como já citado principalmente quanto a tecnologias não foi produtivo a procura em bases científicas. Mas no limite nacional a Sociedade Brasileira de Computação (SBC) mostrou-se grande apoiadora devido aos vários artigos relacionados aos congressos, \textit{workshops} e outros promovidos por essa instituição que puderam ser obtidos.

Dentre o maior tema e também por relação com este, informática na educação que proporcionou material importante para composição dos capítulos presentes dessa dissertação. Quanto pesquisa por conteúdo, no que se refere à encontrar trabalhos que contivessem construção de \textit{software}, teve resultados mínimos. Estes materiais não possuíam em sua estrutura boa descrição de seu desenvolvimento, podendo ter essa situação influenciado no resultado do presente.

Como entusiasta e evangelizador, do compartilhamento de informações e conhecimento em \textit{software} livre, é triste perceber que inclusive em grandes universidades públicas não há aderência a esta filosofia. Lembra-se para que não seja lido aqui como gratuito, pois pode e é por muitas empresas utilizado esse pensamento de maneira comercial gerando bons lucros.

\section{Sugestões para Trabalhos Futuros}

Oferecendo inúmeras possibilidades é desejado que seja uma inspiração para novos trabalhos ligados diretamente a este ou mesmo só por suas possibilidades \textit{offline}. É possível listar as principais levantadas tentando enumerá-las em uma sequência lógica:

\begin{enumerate}

\item Ampliar o escopo das estruturas de algoritmos;
%\item Permitir testes definindo entrada e saída conforme enunciado;
\item Acesso via login e senha com repositório dos códigos;
%\item Interface professor para submissão de lista de exercícios, montadas através de banco de questões colaborativo;
%\item Interface para aluno para receber listas, resolver e submeter ao professor;
%\item Permitir ao professor controle sobre funcionalidade disponíveis como: auto indentação, auto completar palavras reservadas, informações de possíveis erros durante escrita.
\item Desenvolver linguagem que una e venha a substituir a UAL buscando similaridade com Java, mas mantendo a retrocompatibilidade opcional;
\item Permitir flexibilidade para definir as variações de Portugol para que possa ser utilizado por usuários de outros ambientes.

\end{enumerate}
