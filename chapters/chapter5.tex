
\chapter{Considerações Finais}

O objetivo principal deste projeto era apresentar o conceito de aplicação para prática de algoritmos sem instalação, mas permitindo uso \textit{offline}. Neste sentido sua estrutura como tradutor se mostrou desafiadora mesmo após a opção por componentes para montagem de editor e analisadores. Por outro lado, mesmo recente a tecnologia de tratamento atual para acesso desconectado é bem acessível quanto a explicação de sua utilização, considerado como de dificuldade moderada.

Notou-se que há ruptura devido às possibilidades de ferramentas e variações nas suas sintaxes de pseudo-linguagens. Portugol podia ser objetivo quanto à sua estrutura e não há ainda algum trabalho em prol de sua especificação.

Tentativas anteriores foram realizadas para disponibilização de acesso via navegador por ferramenta similar, com uso de \textit{Applet} Java. Porém o iniciante não costuma ter instalado em sua máquina a \textit{Java Virtual Machine} (JVM) e mesmo os navegadores como o Google Chrome tem perdido o suporte a essa tecnologia.

Devido a análise do cenário pode-se ter além do acesso \textit{offline}, a primeira iniciativa de âmbito educacional em ensino superior, considerando este importante nicho dos dispositivos móveis. É gratificante ter como resultado uma aplicação que pode ser utilizada em qualquer dispositivo móvel, já que é raro quem não tenha um smartphone (principalmente entre universitários), mesmo sem se utilizar das linguagens nativas de seus sistemas operacionais.

\section{Relação entre os Objetivos e os Resultados}

O protótipo foi desenvolvido em acordo com o que consta consta definido nos objetivos, sua sintaxe é compatível com UAL, contendo a primeira estrutura ``saída em tela''. Esse desenvolvimento foi realizado de maneira pública no Github e está disponível para livre utilização, constando nas referências deste trabalho.

%Fazer com que sua sintaxe fosse compatível com UAL foi possível, graças a limitação pré-definida inclusive reafirmada no objetivo. Em acordo com os primeiros exercícios do ensino, se tornou como mínimo produto viável dentro da sintaxe pretendida.

Permite o protótipo acesso via navegadores web modernos, mesmo sem conexão com internet e não é necessária qualquer instalação adicional. Esperava-se que para tais fins que fosse mais ampla a compatibilidade, contudo os navegadores compatíveis são os mais populares. Principalmente a Microsoft deveria em breve implementar essa funcionalidade para \textit{Service Worker} em seu navegador (Edge, substituindo o IE) pois é considerado um software promissor que estaria atualizado com as tecnologias, além de que vem pré instalado como navegador padrão em seu popular sistema operacional Windows (na versão 10).

\section{Limitações da Pesquisa}

% NOTE Dificuldade em identificar material
Como já citado principalmente quanto a tecnologias não foi produtivo a procura em bases científicas. Mas no limite nacional a Sociedade Brasileira de Computação (SBC) mostrou-se grande apoiadora devido aos vários artigos relacionados aos congressos, \textit{workshops} e outros promovidos por essa instituição que puderam ser obtidos.

Dentre o maior tema e também por relação com este, informática na educação que proporcionou material importante para composição dos capítulos presentes nesta dissertação. Quanto pesquisa por conteúdo, no que se refere em encontrar trabalhos que contivessem construção de software, teve resultados mínimos. Estes materiais não possuíam em sua estrutura boa descrição de seu desenvolvimento, podendo ter essa situação influenciado no resultado do presente trabalho.

Como entusiasta e evangelizadors do compartilhamento de informações e conhecimento em software livre, é triste perceber que inclusive em grandes universidades públicas não há aderência a esta filosofia, pois de alguns softwares os códigos não foram encontrados. Lembra-se para que não seja lido aqui como gratuito, pois pode e é por muitas empresas utilizado esse pensamento de maneira comercial gerando bons lucros.

\section{Limitações do Trabalho}

Por seguir um processo diferenciado de construção este protótipo limita-se no uso das estruturas de algoritmo, não compreendendo estruturas como vetores, matrizes, funções, abordando exclusivamente algoritmos sem estrutura. São possíveis no protótipo apenas algoritmos simples de saída em tela, pois o foco era a viabilidade e disponibilidade independente de sistema operacional ou instalações, mesmo que \textit{offline}.

Outra limitação foi a não possibilidade de gravar e manter executáveis da saída gerada do algoritmo, entendeu-se que este ponto não essencial para a prática do conhecimento e que uma possível solução no modelo pretendido seria dispendioso e demorado.

Não há a armazenamento dos algoritmos em servidor, por depender de estrutura adicional como hospedagem, além de programação adicional o devido tratamento desta funcionalidade. Ficando por conta do aluno o salvamento, podendo recorrer a algum serviço online ou mesmo memórias flash (pen drive).

%O sistema de tratamento de erros foi explorado até onde complexidade dos algoritmos possíveis dentro do proposto permitiu. Realizados assim neles os testes de erros possíveis.

\section{Sugestões para Trabalhos Futuros}

Oferecendo inúmeras possibilidades é desejado que o presente trabalho seja uma inspiração para novos projetos, ligados diretamente ou mesmo só pela tecnologia para uso \textit{offline} no presente descrita. É possível listar algumas, tentando enumerá-las em uma sequência lógica:

\begin{enumerate}

\item Ampliar o escopo das estruturas de algoritmos;
%\item Permitir testes definindo entrada e saída conforme enunciado;
\item Acesso via login e senha com repositório dos códigos;
%\item Interface professor para submissão de lista de exercícios, montadas através de banco de questões colaborativo;
%\item Interface para aluno para receber listas, resolver e submeter ao professor;
%\item Permitir ao professor controle sobre funcionalidade disponíveis como: auto indentação, auto completar palavras reservadas, informações de possíveis erros durante escrita.
\item Desenvolver linguagem que una e venha a substituir a UAL buscando similaridade com Java, mas mantendo a retrocompatibilidade opcional;
\item Permitir flexibilidade para definir as variações de Portugol para que possa ser utilizado por usuários de outros ambientes.

\end{enumerate}
