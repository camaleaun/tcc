
\chapter{Fundamentação Teórica}

\section{Algoritmos}

\subsection{Ensino Aprendizagem}

% NOTE citar Teoria das Inteligências Múltiplas de Gardner

\section{Pseudoliguagem e Portugol}

\section{Ferramentas Existentes}

\subsection{UAL e Editor UAL}

\subsection{Demais Aplicativos}

\subsection{Plataforma Digitais de Ensino}

\subsubsection{Udacity}

\subsubsection{Codeacademy}

\subsubsection{SoloLearn}

\section{Tecnologias Web}

\subsection{Linguagem de Marcação de Texto}

\subsection{Liguagem de Programação para Web}

\subsubsection{ECMAScript}

\subsection{Navegadores Web}

\subsection{Aprimoramentos para Uso \textit{Offline}}

\section{Compiladores e Interpretadores}

\subsection{Análise Morfológica}

\subsection{Análise Léxica}

\subsection{Tokenização e Parseamento}

\subsection{Análise Sintática}

\subsection{Análise Semantica}
