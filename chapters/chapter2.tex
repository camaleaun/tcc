
\chapter{Fundamentação Teórica}

\section{Algoritmos}

\subsection{Ensino Aprendizagem}

% NOTE citar Teoria das Inteligências Múltiplas de Gardner

\section{Pseudoliguagem e Portugol}

\section{Ferramentas Existentes}

\begin{itemize}

\item Portugol/Plus

\item VisualG

\item G-Portugol

\item Portugol IDE

\item Portugol Studio

\end{itemize}

\subsection{UAL e Editor UAL}

\subsection{Demais Aplicativos}

\subsection{Plataforma Digitais de Ensino}

\subsubsection{Udacity}

\subsubsection{Codeacademy}

\subsubsection{SoloLearn}

\subsection{Programação em Blocos}

\subsubsection{\textit{Blockly}}

\section{\textit{Online Judge}}

Sistema de Apoio a Competições de Programação é a denominação em português de
\textit{Online Judge} por \citeonline{de2004boca}, nomeado como BOCA.
E mais recente foi desenvolvido por acadêmico da URI (Universidade Regional
Integrada do Alto Uruguai e das Missões) com base nesse trabalho anterior o URI
\textit{Judge Online} \cite{tonin2013uri}.

\section{Tecnologias Web}

\subsection{Linguagem de Marcação de Texto}

\subsection{Liguagem de Programação para Web}

\subsubsection{ECMAScript}

\subsection{Navegadores Web}

\subsection{Aprimoramentos para Uso \textit{Offline}}

\section{Compiladores e Interpretadores}

\subsection{Análise Morfológica}

\subsection{Análise Léxica}

\subsection{Tokenização e Parseamento}

\subsection{Análise Sintática}

\subsection{Análise Semantica}
