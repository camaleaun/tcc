
\chapter{Fundamentação Teórica}

\section{Algoritmos}

A última das notas de Ada Lovelace, que foram republicadas em 1953, apresenta os
primeiros conceitos sobre programação, descrevendo um algoritmo. Desde então
estes foram difundidos, sendo utilizados até a atualidade
\cite{santiago2003etal}.

% TODO Com base nos dois parágrafos abaixo, reescrever utilizado como referência de embasamento os autores \cite{santiagodazzi2003}

%Um algoritmo consiste em um procedimento, composto por uma série de passos
%utilizados para resolver problemas computacionais específicos, que a partir do
%processamento comdados de entradas irá gerar dados de saídas (CORMEN et al,
%1999).

%Para efetuar funcionalidade em um algoritmo e verificar a integridade deste é
%necessário testar o algoritmo verificando o conteúdo das variáveis passo a
%passo. Para efetuar esta tarefa costuma-se utilizar o Teste de Mesa. Também
%chamado Teste Exaustivo, executa para cada instrução, uma verificação, e a
%amostragem do conteúdo das variáveis utilizadas no algoritmo, permitindo que o
%programador visualize o comportamento de todo o processo. Isso permite não
%apenas a comprovação do correto funcionamento, mas também detectar e corrigir
%com facilidade eventuais erros.

\subsection{Ensino Aprendizagem}

% NOTE citar Teoria das Inteligências Múltiplas de Gardner

\section{Pseudoliguagem e Portugol}

\section{Ferramentas Existentes}

\begin{itemize}

\item Portugol/Plus

\item Visual Alg

\item G-Portugol

\item Portugol IDE

\item Portugol Studio

\item ASA

\item ATMUF

\item AWTM

\item AMBAP

\item CIFluxProg

\item RAFF

\item SistLog

\item Ambiente SICAS

\item C-Tutur

\item PL-Detective

\end{itemize}

\subsection{UAL e Editor UAL}

\subsection{Demais Aplicativos}

\subsection{Plataforma Digitais de Ensino}

\subsubsection{Udacity}

\subsubsection{Codeacademy}

\subsubsection{SoloLearn}

\subsection{Programação em Blocos}

\subsubsection{\textit{Blockly}}

\section{\textit{Online Judge}}

Sistema de Apoio a Competições de Programação é a denominação em português de
\textit{Online Judge} por \citeonline{campos2004etal}, nomeado como BOCA.
E mais recente foi desenvolvido por acadêmico da URI (Universidade Regional
Integrada do Alto Uruguai e das Missões) com base nesse trabalho anterior o URI
\textit{Judge Online} \cite{tonin2012etal}.

\section{Tecnologias Web}

\subsection{Linguagem de Marcação de Texto}

\subsection{Liguagem de Programação para Web}

\subsubsection{ECMAScript}

\subsection{Navegadores Web}

\subsection{Aprimoramentos para Uso \textit{Offline}}

\section{Compiladores e Interpretadores}

\subsection{Análise Morfológica}

\subsection{Análise Léxica}

\subsection{Tokenização e Parseamento}

\subsection{Análise Sintática}

\subsection{Análise Semantica}
