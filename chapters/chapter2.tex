
\chapter{Fundamentação Teórica}

\section{Algoritmos}

\ifdraft{\color{green}}{}A última das notas de Ada Lovelace, que foram republicadas em 1953\nocite{1253887}, apresenta os
primeiros conceitos sobre programação, descrevendo um algoritmo. Desde então
estes foram difundidos, sendo utilizados até a atualidade
\cite{santiago2003etal}.\color{black}

Conforme \citeonline{medina2006etal} algoritmo é um termo mais amplo, sendo inclusive destinado a outras áreas e finalidades além da programação. As encontradas na literatura são interpretativas porém todas chegam ao mesmo resultado, um conjunto de instruções para a resolução de uma situação. Algumas afirmações determinam instruções detalhadas ou mesmo que tipo de situação, se especificamente um problema ou tarefas em geral.

\section{Linguagem Natural}

Com menor rigidez do que se faz necessário para uma máquina, esses conjuntos também podem ser descritos em uma linguagem humanamente mais natural. Numa conversa ou em uma explicação de rota de um destino ou uma receita culinária o entendimento ocorrerá. Por depender de formalidade e buscando essa aproximação é que o idioma estruturado é utilizado. Em nosso caso o português estruturado, mas há também o inglês estruturado e certamente outras \ifdraft{\cite{citar}}{\cite{medina2006etal}}.

\section{Programas de Computador}

Escritos em linguagens de programação normalmente de níveis elevados passam por camadas que são traspostas em sua maioria por um processo de compilação. Esse processo converte o código fonte escrito na variação padronizada de código para um linguagem binária (apenas zeros e uns) para comunicação direta o equipamento. Existem também linguagens de níveis mais baixos, porém de menor utilização \ifdraft{\cite{citar}}{\cite{medina2006etal}}.

\section{Pseudo código ou linguagem}

Visando ser uma ponte entre a linguagem natural e as linguagens de programação temos as pseudo-linguagem. Notadamente Portugol é evidênciada como a solução, porém não há uma podronização existindo inúmeras variações, no entando com muitas semelhanção entre si \ifdraft{\cite{citar}}{\cite{medina2006etal}}.

% TODO Com base nos dois parágrafos abaixo, reescrever utilizado como referência de embasamento os autores \cite{santiagodazzi2003}

%Um algoritmo consiste em um procedimento, composto por uma série de passos
%utilizados para resolver problemas computacionais específicos, que a partir do
%processamento comdados de entradas irá gerar dados de saídas (CORMEN et al,
%1999).

%Para efetuar funcionalidade em um algoritmo e verificar a integridade deste é
%necessário testar o algoritmo verificando o conteúdo das variáveis passo a
%passo. Para efetuar esta tarefa costuma-se utilizar o Teste de Mesa. Também
%chamado Teste Exaustivo, executa para cada instrução, uma verificação, e a
%amostragem do conteúdo das variáveis utilizadas no algoritmo, permitindo que o
%programador visualize o comportamento de todo o processo. Isso permite não
%apenas a comprovação do correto funcionamento, mas também detectar e corrigir
%com facilidade eventuais erros.

\ifdraft{}{
%\subsection{Ensino Aprendizagem}

% NOTE citar Teoria das Inteligências Múltiplas de Gardner

%\section{Pseudoliguagem e Portugol}

\section{Ferramentas Existentes}

\begin{itemize}

\item Portugol/Plus

\item Visual Alg

\item G-Portugol

\item Portugol IDE

\item Portugol Studio

\item ASA

\item ATMUF

\item AWTM

\item AMBAP

\item CIFluxProg

\item RAFF

\item SistLog

\item Ambiente SICAS

\item C-Tutur

\item PL-Detective

\end{itemize}

\subsection{UAL e Editor UAL}

\subsection{Demais Aplicativos}

%\subsection{Plataforma Digitais de Ensino}
\subsection{Ambiente Virtual de Aprendizagem - AVA}

% NOTE Interactive Learning - Aprendizado Interativo {tonin2012etal}

\subsubsection{Udacity}

\subsubsection{Codeacademy}

\subsubsection{SoloLearn}

\subsubsection{Code School}

\subsubsection{Kan Academy}

\subsection{Programação em Blocos}

\subsubsection{\textit{Blockly}}

\section{\textit{Online Judge}}

Sistema de Apoio a Competições de Programação é a denominação em português de
\textit{Online Judge} por \citeonline{campos2004etal}, nomeado como BOCA.
E mais recente foi desenvolvido por acadêmico da URI (Universidade Regional
Integrada do Alto Uruguai e das Missões) com base nesse trabalho anterior o URI
\textit{Judge Online} \cite{tonin2012etal}.

\section{Tecnologias Web}

\subsection{Linguagem de Marcação de Texto}

\subsection{Liguagem de Programação para Web}

\subsubsection{ECMAScript}

\subsubsection{WebAssembly}

\subsection{Navegadores Web}

\subsection{Aprimoramentos para Uso \textit{Offline}}

\section{Compiladores e Interpretadores}

%Um compilador traduz um programa descrito em uma linguagem de alto nível, mais
%adequada aos seres humanos, para os códigos em uma linguagem de máquina, que
%podem ser executados por um processador. Grosso modo, o processo de
%compilação pode ser descrito como contendo as seguintes etapas: análise léxica, análise
%sintática, análise semântica e geração de código.\cite{barbosa2009etal}

\subsection{Análise Morfológica}

\subsection{Análise Léxica}

\subsection{Tokenização e Parseamento}

\subsection{Análise Sintática}

\subsection{Análise Semantica}
}
