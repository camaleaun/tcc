\ifdraft{\color{green}}{}\chapter{Métodos de Pesquisa}

Segundo \citeonline{prodanov2013etal} a pesquisa aplicada tem como propósito produzir compreensão sobre determinada questão objetivando aplicação prática, assim esta pesquisa classifica-se como aplicada, pois busca resolver o problema de acessibilidade ao softwares de prática no aprendizado de algoritmo.

\section{Classificação das Pesquisa}

Quanto aos objetivos a pesquisa qualifica-se do tipo exploratória, procurando conhecer as ferramentas existentes para aprendizagem de algoritmo.

A fase exploratória da pesquisa tem como função  promover mais conhecimentos sobre o assunto \cite{prodanov2013etal}.

\section{Método Científico}

Quanto aos procedimentos, a pesquisa realizou-se através de levantamento bibliográfico, realizada através de material já publicado \cite{prodanov2013etal}. Apurando artigos científicos, teses e monografias disponíveis publicamente na internet, principalmente via Google Acadêmico.

Inicialmente a pesquisa bibliográfica foi instituída tendo como meta os últimos 10 anos, mas como foram encontradas publicações relevantes vários anos anteriores esse valor foi alterado para o dobro, 20 anos. Nenhuma relevância encontrada no Brasil anterior à 1998 passando esse a ser o limiar, com dois trabalhos neste ano.

\section{Universo da Pesquisa}

Segundo \citeonline{prodanov2013etal} o universo da pesquisa caracteriza-se pela similaridade de  características em uma população. Desta forma esta pesquisa está inserida no universo de desenvolvimento de software.

\section{População e Amostra}

Amostra da pesquisa define-se pela parcela de individuos inseridas no universo da pesquisa. \cite{prodanov2013etal}. Assim sendo, a amostra da pesquisa tem como alvo os softwares desenvolvidos para prática de exercícios durante o aprendizado de algoritmos.\color{black}

\ifdraft{}{
%\section{Estudo da Ferramenta EditorUAL}
%
%\section{Estudo da Outras Ferramenta Relacionadas}
%
%\section{Definição de Requisitos}
%
%\section{Desenvolvimento}
%
%\subsection{Interface Gráfica}
%
%\subsection{Módulo de Classificação}
%
%\subsection{Módulo de Interpretação}
}
