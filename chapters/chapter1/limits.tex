Limitou-se o trabalho numa imposição que vem preservar que o tempo disposto seja viável para sua execução.

Por seguir um processo bem diferenciado de construção optou-se por sua mínima execução. Limita-se no uso das  estruturas do algoritmo, não compreendendo estruturas como vetores, matrizes, funções o trabalho aborda exclusivamente algoritmos sem estrutura. Permitindo apenas algorítmos simples de saída em tela, pois o focos é a disponibilidade independente de sistema operacional ou instalações.

Outra limitação foi a não possibilidade de gravar e manter executáveis da saída gerada do algoritmo. Devido ao entendimento que é ponto não essencial para a prática do conhecimento e que uma possível solução no modelo pretendido seria dispendioso e demorado.

Devido a depender de estrutura adicional como servidores e hospedagem além de programação adicional desta interface também não há a gravação automática dos códigos para acesso irrestrito. Cabendo ao salvamento e abertura pela máquina. Levou-se em conta que há vários serviços para tal atividade com no caso cabendo ao usuário.

O sistema de tratamento de erros foi explorado até onde complexidade dos algoritmos possíveis dentro do proposto permitiu. Realizados assim neles os testes de erros possíveis.
