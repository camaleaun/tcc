% NOTE JOSELI: O que é programação
% NOTE Diferenciar programação de algoritmos
A base da programação é ensinar máquina a realizar determinada tarefa. Para isso
pode-se dizer que independente do paradigma utilizado no processo de
desenvolvimento ele envolve algoritmos.

% NOTE JOSELI: Algoritmos
Algoritmos podem não necessariamente serem computacionais, já que são um
conjunto de passos finitos. Analogias simplistas são os ingredientes e modo de
preparo de uma receita ou instruções para uma tarefa do cotidiano como a troca
de um pneu furado em um automóvel.

% NOTE JOSELI: Ferramentas
Porém se tratando de computação deve ter maior formalidade, seguindo uma linguagem
pois diferente do ser humano uma máquina não é tão interpretativa.
Desconsideradas aqui inteligências artificiais que são desenvolvidas por
algoritmos para terem  comportamento de compreensão humana. Isto pode ser
analogamente algo como não entender idioma estrangeiro e as instruções estarem
nesta linguagem. Porém em informática isso tem que ser mais especificado com
palavras ou caracteres predeterminados de início e fim, de blocos, repetições,
etc.

% NOTE JOSELI: Portugol e UAL
Sendo o berço mais popular da informática os Estados Unidos da América e seus
criadores em muitos casos dessa origem ou de países com mesmo idioma as
primeiras linguagens de computação e a maioria delas são em inglês.

% NOTE JOSELI: Portugol e UAL
Isso muda quando falamos do Portugol que possui variações, mas resumidamente
sua origem vem como uma tradução de uma linguagem de programação simples e comum,
com seus termos no idioma nativo do Brasil, devido a colonização
portuguesa.

% NOTE JOSELI: Portugol e UAL
Tendo como foco o aprendizado de algoritmos para uma boa fundamentação de
futuros programadores ou profissionais da área, e muitas vezes a facilidade com
o inglês não é algo comum em nosso país, a linguagem Portugol se torna um
artifício vantajoso.

% TODO Analisar necessidade de reecrista, remoção, relocação
% NOTE JOSELI: Editor Web
O que se propõe aqui em ter ferramenta de aprendizado composta de editor de
texto para a escrita dos algoritmos e visualizador do resultado do mesmo.
E que não seja necessária instalação, seja facilmente acessível como qualquer
outra página web em computadores ou dispositivos móveis e mesmo assim permaneça
acessível mesmo sem conexão disponível com a internet.

% TODO Analisar necessidade de reecrista, remoção, relocação
% NOTE JOSELI: Editor Web
Portugol, que tem o propósito de facilitar o estudo da lógica de programação vem
Este projeto visa demonstrar que um novo interpretador para a pseudolinguagem
ao encontro como solução para muitas dificuldades atuais no âmbito de
aprendizagem acadêmica. Para tal, seguiu-se a ideia de implementar este
interpretador, onde haverá a preocupação em desenvolvê-lo como protótipo em dois
módulos: Interpretador e Interface Homem-Máquina além de sugerir um Sistema de
Informação para o Professor. Essa ferramenta é um interpretador para a
pseudolinguagem Portugol, gratuito e com uma interface adequada.
