\ifdraft{\color{green}}{}A base da programação é ensinar máquina a realizar determinada tarefa, para que isso seja possível é necessário que o usuário comunique-se com a máquina, numa linguagem que ela entenda, a linguagem de programação. A construção da linguagem de programação envolve ações definidas em passos que se deseja que a máquina execute, a estes passos damos o nome de algoritmos. Para isso pode-se dizer que independente do paradigma utilizado no processo de desenvolvimento ele envolve algoritmos \cite{medeiros2015}.

Algoritmos podem não necessariamente serem computacionais, já que são um conjunto de passos finitos. Analogias simplistas são os ingredientes e modo de preparo de uma receita ou instruções para uma tarefa do cotidiano como a troca de um pneu furado em um automóvel \cite{medina2006etal}. Porém se tratando de computação deve ter maior formalidade, seguindo uma linguagem pois diferente do ser humano uma máquina não é tão interpretativa. Desconsideradas aqui inteligências artificiais que são desenvolvidas por algoritmos para terem  comportamento de compreensão humana. Isto pode ser analogamente algo como não entender idioma estrangeiro e as instruções estarem nesta linguagem. Porém em informática isso tem que ser mais especificado com palavras ou caracteres predeterminados de início e fim, de blocos, repetições, etc \cite{dershem1990etal}.

Sendo o berço mais popular da informática os Estados Unidos da América e seus criadores em muitos casos dessa origem ou de países com mesmo idioma as primeiras linguagens de computação e a maioria delas são em inglês \cite{sebesta2009}.

Isso muda quando falamos do Portugol que possui variações, mas resumidamente sua origem vem como uma tradução de uma linguagem de programação simples e comum, com seus termos no idioma nativo do Brasil.

Tendo como foco o aprendizado de algoritmos para uma boa fundamentação de futuros programadores ou profissionais da área, e muitas vezes a facilidade com o inglês não é algo comum em nosso país, a linguagem Portugol se torna um artifício vantajoso \cite{jesus2004etal}.\color{black}
