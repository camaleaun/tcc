O presente trabalho está estruturado em cinco capítulos que abordam o Processo de Ensino e Aprendizagem comum na atualidade e as Ferramentas Utilizadas, detalhes sobre o ensino com a utilização de Processamento de Linguagem Natural em Algoritmos. Após o primeiro o anterior introdutório o capítulo dois descreve principais termos tem sua teoria descrita além de apresentar as principais ferramentas utilizadas no ensino e prática de lógica de programação.

Os capítulos seguintes apresentam no três a metodologia utilizada e no quatro aplicação em seu desenvolvimento até o seu resultado, apresentando alguns temos pertinentes somente ao processo. Por fim o capítulo cinco conclui-se com as considerações finais, recomendações, sugestões e trabalhos futuros, podendo ser pelo autor ou outros interessados.
