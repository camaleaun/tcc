No processo de ensino ao invés de abstração em papel, utilizar prática em software auxilia e muito o aproveitamento na aprendizagem. Em algoritmos é necessário o pensamento lógico, em alguns casos o aluno vêm com deficiência inclusive em lógica geral. ``O Aprendizado de algoritmos não é uma tarefa muito fácil, só se consegue através de muitos exercícios'' \cite[p.~1]{lopes2002etal}. Para esse ensino é essencial conhecer dois conceitos principais: lógica de programação e algoritmos. Primeiramente é apresentado um paralelo de situações do cotidiano com algoritmos, ao efetuar comparações de instruções passo a passo realizadas diariamente, como fritar um ovo ou trocar um pneu de automóvel.

Após o entendimento básico da lógica, passa-se então para o uso de uma ferramenta computacional para a simulação destes algoritmos, aplicando descrição na linguagem estruturada em português. Em alguns casos, utiliza-se de linguagens em inglês, como por exemplo, Pascal. Há professores que preferem usar pseudo linguagem em português somente ``em papel'', iniciando a codificação diretamente em linguagem C. Tem-se inclusive, também ``em papel'', de analisar e interpretar os passos do algoritmo para saber quais serão os valores em memória e a saída na tela.

Simultaneamente, com a prática escrita, sem software específico, pode-se simular os mesmos algoritmos em interpretador ou compilador. Sendo em um interpretador sua construção dividida em partes, cada uma com uma função específica. Geralmente identifica-se: o analisador léxico, o analisador sintático, o analisador semântico, e o resultado de saída. Caso seja um compilador há, ao invés de resultado de saída, o gerador de código, que transforma o programa em linguagem de máquina composta de 0s e 1s \cite{delamaro2004}.

Diante do exposto, este trabalho justifica-se pelo empenho em facilitar o processo de aprendizagem de algoritmos, desenvolvendo um programa em navegador via Internet, fazendo com que seja desnecessária a instalação do programa,  podendo ser acessado a partir de qualquer computador, com o objetivo que os exercícios sejam resolvidos com maior praticidade.\color{black}
