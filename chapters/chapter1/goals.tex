\subsection{Objetivo Geral}

%Apresentar o conceito de ferramenta computadorizada para se tornar referência
%em criação, edição, visualização e interpretação de algoritmos em
%pseudo linguagem, como alternativa às existentes. Sendo seu diferencial o acesso
%nativo a partir de navegadores \textit{Web} modernos. Inclusive acessível
%quando uma conexão com a internet esteja indisponível.
Desenvolver um protótipo de ferramenta computadorizada para prática de algoritmos, podendo realizar a criação, edição, visualização e interpretação de algoritmos em pseudo linguagem Portugol. Seu acesso deverá ser nativo a partir de navegadores web modernos, inclusive acessível quando esteja indisponível uma conexão com a internet.
% NOTE \footnote{Desde que haja ao menos um acesso anterior.}

\subsection{Objetivos Específicos}

Visando atingir o objetivo geral em sua totalidade lista-se a seguir objetivos
específicos de modo sucinto:

\begin{itemize}
\setlength\itemsep{0em}

%\item Desenvolver o protótipo, sendo seu código fonte aberto e
%disponível para que outros possam manter, ampliar e evoluir a aplicação;
\item Desenvolver o protótipo;

%\item Fazê-lo compatível com a sintaxe desenvolvida na Universidade Estácio
%de Sá (UNESA) a UNESA \textit{Algorithmic Language} (UAL). E mínimo produto
%viável para a primeira lição de ensino aprendizagem;
\item Fazê-lo compatível com a sintaxe atualmente utilizada;

%\item Permitir o acesso via navegadores \textit{Web} modernos, sem
%necessidade de instalações complementares;
\item Permitir o acesso nativo via navegadores web modernos;

% Para isso utilizar HTML5 que pode ser descrito como o uso de somente linguagem de programação ECMAScript (popularmente Javascript), e para inferface gráfica essencialmente a combinação de HTML (abreviação para a expressão inglesa HyperText Markup Language, que significa Linguagem de Marcação de Hipertexto) e de estilização em Cascading Style Sheets (CSS) em suas versões recomendáveis pelo World Wide Web Consortium (W3C).

%\item Aplicar aprimoramentos introduzidos na versão 5 da
%\textit{HyperText Markup Language} (HTML) para uso sem conexão com a
%internet.
\item Aplicar funcionalidade para para uso sem conexão com a internet.

\end{itemize}
