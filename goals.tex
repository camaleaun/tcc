\subsection{Objetivo Geral}

Apresentar o conceito de ferramenta computadorizada para se tornar referência
em criação, edição, visualização e interpretação de algoritmos em
pseudolinguagem, como alternativa às existentes. Sendo seu diferencial o acesso
a partir de navegadores Web por ser em HTML5. Inclusive acessível quando uma
conexão com a internet esteja indisponível.
%\footnote{Desde que haja ao menos um acesso anterior.}

\subsection{Objetivos Específicos}

Visando atingir o objetivo geral em sua totalidade lista-se a seguir objetivos
específicos de modo sucinto:

\begin{itemize}
  %\setlength\itemsep{0em}
  \item Desenvolver o protótipo, sendo seu código fonte aberto e
  disponível para que outros possam manter, ampliar e evoluir a aplicação;
  \item Fazê-lo compatível com sintaxe UAL e mínimo produto viável para as
  primeiras lições de ensino aprendizagem;
  \item Permitir o acesso via navegadores web modernos, sem necessidade de
  instalações complementares;
  % Para isso utilizar HTML5 que pode ser descrito como o uso de somente linguagem de programação ECMAScript (popularmente Javascript), e para inferface gráfica essencialmente a combinação de HTML (abreviação para a expressão inglesa HyperText Markup Language, que significa Linguagem de Marcação de Hipertexto) e de estilização em Cascading Style Sheets (CSS) em suas versões recomendáveis pelo World Wide Web Consortium (W3C).
  \item Aplicar aprimoramentos introduzidos na versão 5 do HTML
  (\textit{HyperText Markup Language}) para uso sem conexão com a internet.
\end{itemize}
