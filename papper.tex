%% abtex2-modelo-trabalho-academico.tex, v-1.9.2 laurocesar
%% Copyright 2012-2014 by abnTeX2 group at http://abntex2.googlecode.com/
%%
%% This work may be distributed and/or modified under the
%% conditions of the LaTeX Project Public License, either version 1.3
%% of this license or (at your option) any later version.
%% The latest version of this license is in
%%   http://www.latex-project.org/lppl.txt
%% and version 1.3 or later is part of all distributions of LaTeX
%% version 2005/12/01 or later .
%%
%% This work has the LPPL maintenance status `maintained'.
%%
%% The Current Maintainer of this work is the abnTeX2 team, led
%% by Lauro César Araujo. Further information are available on
%% http://abntex2.googlecode.com/
%%
%% This work consists of the files abntex2-modelo-trabalho-academico.tex,
%% abntex2-modelo-include-comandos and abntex2-modelo-references.bib
%%

% ------------------------------------------------------------------------
% ------------------------------------------------------------------------
% abnTeX2: Modelo de Trabalho Academico (tese de doutorado, dissertacao de
% mestrado e trabalhos monograficos em geral) em conformidade com
% ABNT NBR 14724:2011: Informacao e documentacao - Trabalhos academicos -
% Apresentacao
% ------------------------------------------------------------------------
% ------------------------------------------------------------------------

\documentclass[
	% -- opções da classe memoir --
	12pt,				% tamanho da fonte
	openright,			% capítulos começam em pág ímpar (insere página vazia caso preciso)
	oneside,			% para impressão sem verso e anverso. Oposto a twoside
	a4paper,			% tamanho do papel.
	% -- opções da classe abntex2 --
	chapter=TITLE,		% títulos de capítulos convertidos em letras maiúsculas
	%section=TITLE,		% títulos de seções convertidos em letras maiúsculas
	%subsection=TITLE,	% títulos de subseções convertidos em letras maiúsculas
	%subsubsection=TITLE,% títulos de subsubseções convertidos em letras maiúsculas
	% -- opções do pacote babel --
	english,			% idioma adicional para hifenização
%	french,				% idioma adicional para hifenização
%	spanish,			% idioma adicional para hifenização
	brazil				% o último idioma é o principal do documento
	]{abntex2-udesc}

% ---
% Pacotes básicos
% ---
\usepackage{lmodern}			% Usa a fonte Latin Modern
\usepackage[T1]{fontenc}		% Selecao de codigos de fonte.
\usepackage[utf8]{inputenc}		% Codificacao do documento (conversão automática dos acentos)
\usepackage{lastpage}			% Usado pela Ficha catalográfica
\usepackage{indentfirst}		% Indenta o primeiro parágrafo de cada seção.
\usepackage{color}				% Controle das cores
\usepackage{graphicx}			% Inclusão de gráficos
\usepackage{microtype} 			% para melhorias de justificação
\usepackage{mathptmx}
\usepackage[singlelinecheck=false]{caption}
% ---

% ---
% Pacotes de citações
% ---
%\usepackage[brazilian,hyperpageref]{backref}	 % Paginas com as citações na bibl
\usepackage[alf,abnt-emphasize=bf,abnt-etal-cite=2]{abntex2cite}	% Citações padrão ABNT
%,abnt-full-initials=yes
%\citeoption{abnt-and-type=&}

% ---
% Pacotes de cor em tabela
% ---
%\usepackage[table]{xcolor}
%\usepackage{array,ragged2e}

% ---
% Pacotes de algoritmos
% ---
\usepackage{listings}

% ---
% Pacotes de simbolos
% ---
\usepackage{pifont}

% ---
% CONFIGURAÇÕES DE PACOTES
% ---

% ---
% Configurações do pacote backref
% Usado sem a opção hyperpageref de backref
%\renewcommand{\backrefpagesname}{Citado na(s) página(s):~}
% Texto padrão antes do número das páginas
%\renewcommand{\backref}{}
% Define os textos da citação
%\renewcommand*{\backrefalt}[4]{
%	\ifcase #1 %
%		Nenhuma citação no texto.%
%	\or
%		Citado na página #2.%
%	\else
%		Citado #1 vezes nas páginas #2.%
%	\fi}%
% ---% ---
% Configurações do pacote listings
\lstdefinelanguage{portugol}
  {morekeywords={Algoritmo,Var,inteiro,Inicio,escreva,Fim}
}
\lstset{
  breakatwhitespace=false,
  breaklines=true,
  escapeinside={\%*}{*)},
  keywordstyle=\bfseries\color{black},
  language=portugol,
  showstringspaces=false,
}
% ---% ---

% ---
% Informações de dados para CAPA e FOLHA DE ROSTO
% ---
\titulo{Interpretador de Algoritmos Portugol em HTML5}
\autor{Gilberto Tavares}
\cidade{São Bento do Sul}
\uf{SC}
\local{\inserecidade, \insereuf}
\data{2016}
\orientador{Dr. Luiz Cláudio Dalmolin}
\instituicao{Universidade do Estado de Santa Catarina - UDESC}
\centro{Centro de Educação do Planalto Norte - CEPLAN}
\curso{Bacharelado em Sistemas de Informação}
\tipotrabalho{Trabalho de Conclusão de Curso}
% O preambulo deve conter o tipo do trabalho, o objetivo,
% o nome da instituição e a área de concentração
\preambulo{Trabalho de Conclusão apresentado ao Curso de Bacharelado em Sistemas
de Informação, da Universidade do Estado de Santa Catarina, como requisito para
a obtenção do grau de Bacharel em Sistemas de Informação}
% ---

% ---
% Configurações de aparência do PDF final

% alterando o aspecto da cor azul
%\definecolor{blue}{RGB}{41,5,195}
\definecolor{blue}{rgb}{0,0,0}

% informações do PDF
\makeatletter
\hypersetup{
     	%pagebackref=true,
		pdftitle={\@title},
		pdfauthor={\@author},
    	pdfsubject={\imprimirpreambulo},
	    pdfcreator={LaTeX with abnTeX2},
		pdfkeywords={lógica da programação}{algoritmos}{portugol}{interpretador}{html5},
		colorlinks=true,       		% false: boxed links; true: colored links
    	linkcolor=black,          	% color of internal links
    	citecolor=blue,        		% color of links to bibliography
    	filecolor=magenta,      		% color of file links
		urlcolor=blue,
		bookmarksdepth=4
}
\makeatother
% ---

% ---
% Espaçamentos entre linhas e parágrafos
% ---

% O tamanho do parágrafo é dado por:
\setlength{\parindent}{1.3cm}

% Controle do espaçamento entre um parágrafo e outro:
\setlength{\parskip}{0.2cm}  % tente também \onelineskip

% ---
% compila o indice
% ---
\makeindex
% ---

% ----
% Início do documento
% ----
\begin{document}
% Retira espaço extra obsoleto entre as frases.
\frenchspacing

% ----------------------------------------------------------
% ELEMENTOS PRÉ-TEXTUAIS
% ----------------------------------------------------------
% \pretextual

% ---
% Capa
% ---
% TODO PRINT Descomentar
%\imprimircapa
% ---

% ---
% Folha de rosto
% (o * indica que haverá a ficha bibliográfica)
% ---
%\imprimirfolhaderosto*
% TODO PRINT Descomentar
%\imprimirfolhaderosto
% ---

% Folha de Aprovação
% TODO PRINT Descomentar
%
% ---
% Inserir folha de aprovação
% ---

% Isto é um exemplo de Folha de aprovação, elemento obrigatório da NBR
% 14724/2011 (seção 4.2.1.3). Você pode utilizar este modelo até a aprovação
% do trabalho. Após isso, substitua todo o conteúdo deste arquivo por uma
% imagem da página assinada pela banca com o comando abaixo:
%
% \includepdf{folhadeaprovacao_final.pdf}
%
\begin{folhadeaprovacao}

  \begin{center}
    {\ABNTEXchapterfont\bfseries\MakeTextUppercase{\imprimirautor}}

    \vspace*{\fill}\vspace*{\fill}
    \begin{center}
      \ABNTEXchapterfont\bfseries\MakeTextUppercase{\imprimirtitulo}
    \end{center}
    \vspace*{\fill}

   \end{center}

  \noindent
  \imprimirpreambulo

  \vspace*{\fill}

  \noindent
  {\bfseries Banca Examinadora}

  \vspace*{\ABNTEXsignskip}

  \setlist[description]{font=\normalfont}

  \begin{description}[labelindent=0pt,labelwidth=4cm] %label={\textnormal},
    \item[Orientador:]
    \assinaturaorientador{\textbf{\href{http://lattes.cnpq.br/7393306373465290}{\imprimirorientador} \\ Universidade do Vale do Itajaí}}
  \end{description}

  \setlist[description]{font=\bfseries}

  \begin{description}[labelindent=0pt,labelwidth=4cm] %label={\textnormal},
    \item[Membros:]
    \vspace*{\ABNTEXsignskip}
    \assinaturaorientador{\textbf{\href{http://lattes.cnpq.br/3985354928735296}{Dr. Nilson Ribeiro Modro} \\ Universidade Federal de Santa Catarina}}
    \item[]
    \vspace*{\ABNTEXsignskip}
    \assinaturaorientador{\textbf{\href{http://lattes.cnpq.br/3317225251677148}{Ma. Nelcimar Ribeiro Modro} \\ Universidade Federal do Paraná}}
  \end{description}

   \begin{center}
    \vspace*{0.5cm}
    \inserecidade~\insereuf, 29/11/2016
    \vspace*{1cm}
  \end{center}

\end{folhadeaprovacao}
% ---


% Dedicatória, Agradecimentos, Epígrafe - Elementos Pré textuais opcionais
% TODO PRINT Descomentar
%
% Dedicatória

% ---
% Dedicatória
% ---
\begin{dedicatoria}
   \vspace*{\fill}
   \hspace{.45\textwidth}
   \begin{minipage}{.5\textwidth}
      \SingleSpacing
Dedico este aos meus familiares falecidos:
% Talvez reticências ...

Meu irmão que sempre apoiou minha formação
superior e carreira até oferecendo certa
colaboração financeira.

E ainda mais recente meu pai que financiou
meu pré ingresso e não pude compartilhar
com ele esta conquista.
   \end{minipage}
	\vspace*{\fill}
\end{dedicatoria}
% ---


% ---
% Agradecimentos
% ---
\begin{agradecimentos}

A minha mãe me apoiou em minha decisão de cursar faculdade noutra cidade. E sempre que possível (mesmo com dificuldade) ajudou como pode: com móveis para meu primeiro quarto, algumas compras de mercado e algum dinheiro quando ficava mais crítico. Inclusive isso tudo pode ter sido um dos motivos da decisão dela em vender o imóvel em que morávamos.

A toda a família Lischka que me acolheu inicialmente como hóspede na residência do casal Arnaldo e Jacira (prima da minha mãe). Foi também na empresa familiar deles, juntos com seus filhos, que tive meu primeiro emprego na cidade e me foi cedido para moradia um quartinho e dependências junto à empresa.

Ao meu irmão que me empregou as sábados e minhas cunhadas que me hospedavam, além de amigos. A todos os motoristas que me deram carona, economizando assim a passagem para que eu pudesse voltar com uma grana a mais do final de semana.

A todos os meus amigos de moradia que tiveram que me aguentar nas repúblicas que habitei, especialmente à mais recentemente extinta ``Mansão Amarela''.

A Joseli que sua experiência passando pelo mesma situação de fim de curso foi de grande ajuda. Também seu apoio e motivação, bem como dos meu amigos de infância José e Rafael que fizeram o mesmo.

A toda a experiência social, esportiva e de conheciento que as confraternizações com outros academicos, Jiudesc e os vários Latinoware proporcionaram.

As empresas Hardline e Xthor que me ofereceram estágio e emprego respectivamente. Principalmente a segunda, na qual todo aprendizado e parceria com o Richard levou ao meu crescimento profissional, acadêmico e pessoal.

Sem esquecer é claro das minhas garrafas térmicas, que mantiveram tanto café quentinho quanto suco de guaraná gelado para embalar as noites dedicadas a esta realização.


\end{agradecimentos}
% ---

% Epígrafe

% ---
% Epígrafe
% ---
\begin{epigrafe}
    \vspace*{\fill}
   \hspace{.45\textwidth}
   \begin{minipage}{.5\textwidth}
      \SingleSpacing

% No idioma de origem
% ``Most good programmers do programming not
% because they expect to get paid or get adulation
% by the public, but because it is fun to program..''

% Tradução original
% ``A maioria dos bons programadores programam não
% porque esperam ser pagos ou adulados pelo público,
% mas porque é divertido programar.''

{\small``A maioria dos bons programadores programa
não por esperar ser pago ou adulado pelo público,
mas porque é divertido programar.''  (tradução nossa)

\vspace{\onelineskip}

Linus Torvalds}
   \end{minipage}
	\vspace*{\fill}
\end{epigrafe}
% ---



% Resumos (em português e pelo menos em inglês)
% TODO PRINT Descomentar
%
% ---
% RESUMOS
% ---

% resumo em português
%MODRO: Resumo Monografia
%Qual o problema? Qual a dificuldade atual?
%Porque da solução?
%Como desenvolvido?
%Diagramas modelagem?
%Tempo de desenvolvimento?
%Processo de aprendizado?
%Ferramentas utilizadas?
%Market share?
%Funcionalidades?
%Onde testado?
%Onde utilizado?
%Casos de uso? Opiniões?
%Documentação / Projeto?
\begin{resumo}
Trata-se do estudo das opções para prática computacional no processo de ensino e aprendizagem, sugerindo um conceito de ferramenta que diferente das existentes, pois permite acesso via navegador de internet sem instalação ou download adicional. Vários conceitos são apresentados sobre algoritmos como: sua estrutura e seus tradutores, sejam eles compiladores ou interpretadores. Mas para o foco do trabalho apenas algoritmo com simples impressão em tela foi implementado, permitindo demonstrar seu diferencial que além de acesso nativo via navegador o pode ser utilizado sem conexão com a internet. Sendo o problema a atual dificuldade de iniciar a execução do programa para a prática durante o aprendizado de algoritmos, vem se propor esta solução para acesso simplifica independente da plataforma ou dispositivo do aluno. Foi desenvolvido utilizando da ferramenta utilizada nesta instituição por alguns professores como modelo, ela foi repensada modo minimalista quanto sua interface, desse pensamento o diagrama de caso de uso foi obtido. Foi necessário um processo de aprendizado sobre compiladores por não termos neste curso ainda alguma disciplina que aborde este tema. Se de modo ininterrupto pode-se concluir como tempo total de desenvolvimento três meses. Foram utilizadas técnicas de vanguarda para funcionamento em navegadores modernos, dentre essas os últimos padrões da linguagem de programação ECMAScript (popularmente JavaScript). Nessa linguagem foram encontrados componentes de código livre: Acorn e JS-Intrepreter para a análise léxico-sintática e execução semântica respectivamente, além do Ace como editor de texto.

\vspace{\onelineskip}
\textbf{Palavras-chaves}: Lógica da programação. Algoritmos. Portugol. Interpretador. HTML5.
% ensino, aprendizagem
\end{resumo}

% resumo em inglês
%\begin{resumo}[Abstract]
%\begin{otherlanguage*}{english}
%
%\vspace{\onelineskip}
%\textbf{Key-words}: Programming logic. Algorithms. Portugol. Interpreter. HTML5.
%% teaching, learning
%\end{otherlanguage*}
%\end{resumo}


% Lista de Ilustrações
% TODO PRINT Descomentar
%

% ---
% inserir lista de ilustrações
% ---
\pdfbookmark[0]{\listfigurename}{lof}
\listoffigures*
\cleardoublepage
% ---


% Lista de Siglas
% TODO PRINT Descomentar
%
% ---
% inserir lista de tabelas
% ---
\pdfbookmark[0]{\listtablename}{lot}
\listoftables*
\cleardoublepage
% ---


% Lista de Siglas
% TODO PRINT Descomentar
%
% ---
% inserir lista de abreviaturas e siglas
% ---
\begin{siglas}
%  \item[CSS] Hyper Text Markup Language
  \item[HTML] \textit{HyperText Markup Language}
  \item[SW] \textit{Service Workers}
  \item[UAL] UNESA \textit{Algorithmic Language}
  \item[UNESA] Universidade Estácio de Sá
\end{siglas}
% ---


% ---
% inserir o sumario
% ---
\pdfbookmark[0]{\contentsname}{toc}
\tableofcontents*
\cleardoublepage
% ---



% ----------------------------------------------------------
% ELEMENTOS TEXTUAIS
% ----------------------------------------------------------
\textual

% ----------------------------------------------------------
% Introdução (exemplo de capítulo sem numeração, mas presente no Sumário)
% ----------------------------------------------------------
\chapter{Introdução}
%\addcontentsline{toc}{chapter}{Introdução}
% ----------------------------------------------------------

\ifdraft{\color{green}}{}A base da programação é ensinar a máquina realizar determinada tarefa, para que isso seja possível é necessário que o usuário comunique-se com a máquina, numa linguagem que ela entenda, a linguagem de programação. A construção da linguagem de programação envolve ações definidas em passos que se deseja que a máquina execute, a estes passos damos o nome de algoritmos. Para isso pode-se dizer que independente do paradigma utilizado no processo de desenvolvimento ele envolve algoritmos \cite{medeiros2015}.

Algoritmos podem não necessariamente serem computacionais, já que são um conjunto de passos finitos. Analogias simplistas são os ingredientes e modo de preparo de uma receita ou instruções para uma tarefa do cotidiano, como a troca de um pneu furado em um automóvel \cite{medina2006etal}. Porém se tratando de computação deve ter maior formalidade, seguindo uma linguagem pois diferente do ser humano uma máquina não é tão interpretativa. Desconsideradas aqui inteligências artificiais que são desenvolvidas por algoritmos para terem  comportamento de compreensão humana. Isto pode ser analogamente algo como não entender idioma estrangeiro e as instruções estarem nesta linguagem. Porém em informática isso tem que ser mais especificado com palavras ou caracteres predeterminados de início e fim, de blocos, repetições, etc \cite{dershem1990etal}.

Sendo o berço mais popular da informática os Estados Unidos da América e seus criadores em muitos casos dessa origem ou de países com mesmo idioma as primeiras linguagens de computação e a maioria delas são em inglês \cite{sebesta2009}. Isso muda quando falamos do Portugol que possui variações, mas resumidamente sua origem vem como uma tradução de uma linguagem de programação simples e comum, com seus termos no idioma nativo do Brasil.

Tendo como foco o aprendizado de algoritmos para uma boa fundamentação de futuros programadores ou profissionais da área, muitas vezes a facilidade com o inglês não é algo comum em nosso país, a linguagem Portugol se torna um artifício vantajoso \cite{jesus2004etal}.  Neste sentido já é utilizado pela maioria dos professores do Centro de Educação Superior do Planalto Norte (CEPLAN) da Universidade do Estado de Santa Catarina (UDESC) livro texto e software para a prática do aprendizado de algoritmos em Portugol.

Porém quando o software é exclusivo para uso em um único sistema operacional, instalações que podem não serem simples ou mesmo não acessível de outros dispositivos, passa a complicação adicional ao aluno iniciante. Por isso o protótipo propoẽ ser acessível em qualquer navegador web, desde que alinhado com as tecnologias mais recentes. Sem contudo deixar de permitir que possa permanecer a prática de algoritmos quando não houver conexão com a internet.


\section{Objetivos do Trabalho}

\ifdraft{\color{green}}{}\subsection{Objetivo Geral}

Apresentar o conceito de ferramenta computadorizada para se tornar referência
em criação, edição, visualização e interpretação de algoritmos em
pseudolinguagem, como alternativa às existentes. Sendo seu diferencial o acesso
a partir de navegadores Web por ser em HTML5. Inclusive acessível quando uma
conexão com a internet esteja indisponível.
% NOTE \footnote{Desde que haja ao menos um acesso anterior.}

\subsection{Objetivos Específicos}

Visando atingir o objetivo geral em sua totalidade lista-se a seguir objetivos
específicos de modo sucinto:

\begin{itemize}
  %\setlength\itemsep{0em}
  \item Desenvolver o protótipo, sendo seu código fonte aberto e
  disponível para que outros possam manter, ampliar e evoluir a aplicação;
  \item Fazê-lo compatível com sintaxe UAL e mínimo produto viável para as
  primeiras lições de ensino aprendizagem;
  \item Permitir o acesso via navegadores web modernos, sem necessidade de
  instalações complementares;
  % Para isso utilizar HTML5 que pode ser descrito como o uso de somente linguagem de programação ECMAScript (popularmente Javascript), e para inferface gráfica essencialmente a combinação de HTML (abreviação para a expressão inglesa HyperText Markup Language, que significa Linguagem de Marcação de Hipertexto) e de estilização em Cascading Style Sheets (CSS) em suas versões recomendáveis pelo World Wide Web Consortium (W3C).
  \item Aplicar aprimoramentos introduzidos na versão 5 do HTML
  (\textit{HyperText Markup Language}) para uso sem conexão com a internet.
\end{itemize}\color{black}


\section{Justificativa do Trabalho}

\ifdraft{\color{green}}{}Praticar em \textit{software} à abstração em papel auxilia e muito no processo de ensino e aprendizagem, em algoritmos além do pensamento lógico por etapas já pode ser desenvolvido e descrito para o repasse a um computador.
``O Aprendizado de algoritmos não é uma tarefa muito fácil, só se consegue através de muitos exercícios'' \cite[p.~1]{lopes2002etal}.

Para esse ensino é essencial conhecer dois conceitos principais lógica de programação e algoritmos. Primeiramente é apresentado um paralelo de situações do cotidiano com algoritmos, ao efetuar comparações de instruções passo a passo realizadas diariamente, como fritar um ovo ou trocar um pneu de automóvel. A seguir inicia-se a aplicação de linguagem estruturada em português para descrição dos algoritmos. Em alguns casos, utiliza-se de linguagens em inglês, mo por exemplo, Pascal. Há professores que preferem usar pseudo linguagem em português somente ``em papel'', iniciando a codificação diretamente em linguagem C. Tem-se inclusive, também ``em papel'', de analisar e interpretar os passos do algoritmo para saber quais serão os valores em memória e a saída na tela.

Simultaneamente, com a prática escrita, sem software específico, pode-se simular os mesmos algoritmos em interpretador ou compilador. Sendo em um interpretador sua construção dividida em partes, cada uma com uma função específica. Geralmente identifica-se: o analisador léxico, o analisador sintático, o analisador semântico, e o resultado de saída. Caso seja um compilador há, ao invés de resultado de saída, o gerador de código, que transforma o programa em linguagem de máquina composta de 0s e 1s \cite{delamaro2004}.

Diante do exposto, este trabalho justifica-se pelo empenho em facilitar o processo de aprendizagem de algoritmos, desenvolvendo um um programa em navegador via Internet, fazendo com que seja desnecessária a instalação do programa,  podendo ser acessado a partir de qualquer computador, com o objetivo que os exercícios sejam resolvidos com maior praticidade\color{black}


\section{Estrutura do Trabalho}


No presente trabalho apresentar-se-á a forma de desenvolvimento do primeiro
módulo, o interpretador, a interface com o usuário e diretrizes para a
execução do módulo restante que não será executado, e que poderão ser ampliados
a partir desta base de estudo e conhecimentos aplicados no presente projeto.

O restante do projeto está organizado da seguinte forma: no capítulo 2 é
relatada a forma como foi idealizado o desenvolvimento da ferramenta proposta;
no capítulo 3 descreve-se como o módulo interpretador está sendo desenvolvido;
nos capítulos 4 e 5 são apresentadas as diretrizes que serão seguidas para os
desenvolvimentos dos módulos Interface Homem-Máquina e sugestão de Sistema de
Informação para o Professor, respectivamente; no capítulo 5 são apresentados as
considerações finais.


% Capítulo 2

\chapter{Fundamentação Teórica}

\section{Ensino Aprendizagem de Algoritmos em Portugol}

Portugol é uma pseudo linguagem algorítmica muito utilizada na descrição de
algoritmos, destaca-se por usar comandos em português, facilitando o aprendizado
da lógica de programação, e desta forma habituando o iniciante com o formalismo
da programação \cite{118}.

% FIX ME REESCREVER com minhas palavras "PLÁGIO" de {miranda2004}
Segundo \citeonline{manzanooliveira2005}, o portugol pode ser classificado como
uma técnica narrativa denominada de pseudocódigo, ou conhecida também por
português estruturado. Baseia-se em uma PDL - \textit{Program Design Language}
(Linguagem de Projeto de Programação), sendo que sua forma original de escrita é
conhecida como inglês estruturado, muito parecida com a notação da linguagem
PASCAL.

% FIX ME REESCREVER com minhas palavras "PLÁGIO" de {miranda2004}
Ainda segundo \citeonline{manzanooliveira2005}, o portugol é usado como
referência genérica para uma linguagem de projeto de programação, e tem por
finalidade mostrar uma notação para elaboração de algoritmos, que será usada na
definição, criação e desenvolvimento de uma linguagem computacional e sua
documentação.

% FIX ME REESCREVER com minhas palavras "PLÁGIO" de {miranda2004}
A representação de algoritmos em portugol, conforme \citeonline{118}, é rica em
detalhes, como a definição dos tipos das variáveis usadas no algoritmo,
encontrando muita aceitação por se assemelhar em demasia à forma em que são
escritos os programas.

% FIX ME REESCREVER com minhas palavras "PLÁGIO" de {miranda2004}
Comparando este processo aos fluxogramas destaca-se uma vantagem, visto que
segundo \citeonline{manzanooliveira2005}, é mais fácil escrever que desenhar
(na grande maioria dos casos), e a codificação acaba se tornando uma simples
transcrição de palavras chave. Corroborando \citeonline{118} aponta o fato do
Portugol ser uma linguagem simples e permitir o detalhamento dos algoritmos.
A tradução de um algoritmo em Portugol para um programa computacional através
de uma linguagem de programação fácil e clara, o que facilita muito o
ensino/aprendizagem da própria linguagem de programação. Outro destaque, é o
fato de a passagem para qualquer linguagem de programação ser quase imediata,
bastando ao aluno saber as palavras reservadas da respectiva linguagem de
programação.

% FIX ME REESCREVER com minhas palavras "PLÁGIO" de {miranda2004}
\citeonline{118} salienta que o inconveniente dos algoritmos é o fato de não
poderem ser executados no computador, pois o iniciante precisa imaginar a sua
execução, e essa tarefa não é fácil. \citeonline{souzaetal2000} apresenta outros
problemas, tais como a influência da linguagem de programação utilizada no
momento da definição do algoritmo e o fato de não haver utilização de recursos
visuais, o que deixa de estimular um fator importante no aprendizado dos alunos,
que é a visão.

\subsection{O Ensino de Algoritmos}

% FIX ME REESCREVER com minhas palavras "PLÁGIO" de {miranda2004}
O ensino de Algoritmos, em cursos de Computação, tem por objetivo ordenar o
pensamento do aluno, fazendo com que o mesmo aprenda a pensar na mesma sequência
lógica utilizada pelo computador. A importância dessa disciplina será percebida
mais tarde, quando o aluno iniciar o aprendizado em linguagens de programação e
precisar ordenar os passos para resolução de um problema, repassando-os ao
computador, sob forma de comandos \cite{miranda2004}.

% FIX ME REESCREVER com minhas palavras "PLÁGIO" de {miranda2004}
Conforme \citeonline{brookshear2000}, o algoritmo é uma codificação do
raciocínio necessário para resolver determinado problema, e essa capacidade
torna possível a construção de máquinas com comportamento inteligente – sendo
essa inteligência moldada através de software, o que torna a disciplina de
algoritmos essencial ao campo de Computação. O próximo passo seria representar
o algoritmo desenvolvido de uma forma apropriada ao entendimento da máquina ou
do aluno, através de comandos compreensíveis e sem ambiguidade, ou seja,
transportá-lo a uma linguagem de programação.

% FIX ME REESCREVER com minhas palavras "PLÁGIO" de {miranda2004}
\citeonline{003} coloca que o objetivo da disciplina de algoritmos é o aluno
desenvolver habilidades cognitivas que permitam que o mesmo aprenda a resolver
problemas, utilizando-se do computador como ferramenta. Os professores, seguindo
o raciocínio de \citeonline{003}, devem se preocupar com fatores como as
diferentes formas de resolução de um mesmo problema, e os diferentes estilos
cognitivos de um aluno. De acordo com \apudonline{gardner1984}{domingues2003},
cada ser humano possui um conjunto inato de competências intelectuais humanas
chamadas de Inteligências Múltiplas. Dentre estas inteligências, existe a
classificada como Lógico Matemática, chamada também de raciocínio científico ou
indutivo, que é a inteligência mais desenvolvida ou potencialmente mais
desenvolvida nos alunos dos cursos de Computação. \citeonline{003} ainda destaca
que o tipo de inteligência predominante em um indivíduo não irá padronizar seu
raciocínio ou a forma que o mesmo resolve problemas.

% FIX ME REESCREVER com minhas palavras "PLÁGIO" de {miranda2004}
O professor da disciplina de algoritmos, de acordo com \citeonline{003}, deve
conduzir a disciplina de forma que o próprio aluno descubra seu estilo de
raciocínio e forma de solução de problemas, auxiliando o aluno a modelar sua
solução em forma de algoritmo. Conforme \citeonline{menezesnobre2002}, o
professor da disciplina de algoritmos vivencia algumas dificuldades durante o
processo de aprendizagem, destacando-se:

% FIX ME REESCREVER com minhas palavras "PLÁGIO" de {miranda2004}
\begin{itemize}
  \setlength\itemsep{0em}
  \item reconhecer as habilidades inatas de seus alunos no processo de ensino;
  \item apresentar diferenciadas técnicas para solução dos problemas;
  \item trabalhar a capacidade do aluno de buscar soluções e escolha da
estrutura a ser utilizada; e
  \item promover a cooperação e colaboração entre os alunos.
\end{itemize}

% FIX ME REESCREVER com minhas palavras "PLÁGIO" de {miranda2004}
Outro problema seria a dificuldade dos alunos em abstrair do problema o que se
deseja. Normalmente, na disciplina de algoritmos, é dada a teoria, logo após
exemplos e exercícios referentes ao tema abordado. Estes exercícios normalmente
são constituídos de uma parte textual, situando o aluno no problema. A grande
dificuldade encontrada é abstrair deste texto o que se deseja, mais
especificamente, as entradas, o processamento e a saída \cite{miranda2004}.

% FIX ME REESCREVER com minhas palavras "PLÁGIO" de {miranda2004}
\citeonline{castrovcastrojr2002} salienta que a abordagem tradicional, que é o
curso baseado na programação imperativa, tem trazido sérios problemas aos alunos
que nunca programaram. Esta abordagem faz com que o aluno comece a pensar um
problema passo a passo, e isto não é o que é feito no dia a dia.
\citeonline{koliveretal2004} coloca que o despreparo da maior parte dos
estudantes em questão de enfrentar uma disciplina que pretende auxiliá-los no
desenvolvimento de soluções para problemas genéricos, em passos e de maneira
lógica, é um dos fatores para os altos índices de desistência e reprovação.

% FIX ME REESCREVER com minhas palavras "PLÁGIO" de {miranda2004}
\citeonline{koliveretal2004} informa que outro fator que causa dificuldade aos
professores é a heterogeneidade do público. Alguns acadêmicos possuem uma
engenhosidade superior nata, o que faz com que possuam uma facilidade natural
para a elaboração de soluções algorítmicas para problemas. Isso faz com que os
instrutores tenham problemas em questão de definição do programa da disciplina
e grau de complexidade dos problemas propostos.

% FIX ME REESCREVER com minhas palavras "PLÁGIO" de {miranda2004}
Dando sequência aos problemas encontrados, normalmente os professores utilizam
no ensino de algoritmos a pseudolinguagem Portugol, complementada pelo
fluxograma (até parte da disciplina). Parte dos alunos sentem dificuldades em
compreender estas estruturas. Conseguem analisar o problema, retirar dele as
entradas, processamento e saídas solicitadas, mas no momento de transportar para
a pseudolinguagem, sentem dificuldade em compreender de que forma isso deve ser
feito \cite{miranda2004}.

% FIX ME REESCREVER com minhas palavras "PLÁGIO" de {miranda2004}
Outro ponto interessante para ser analisado é a forma como os estudantes chegam
ao terceiro grau, conforme informa \citeonline{koliveretal2004}. Os calouros
normalmente chegam carregando vícios oriundos de práticas mecanicistas
frequentes nos primeiro e segundo graus, que acostuma o estudante a realizar
aplicações de fórmulas matemáticas sem realizar qualquer tipo de análise do
problema. Desta forma, o acadêmico entra na universidade sem conhecer um método
geral de resolução de problemas. Isto faz com que muitos professores da
disciplina possuam uma expectativa equivocada relativa ao conhecimento do aluno,
esperando que o mesmo possua habilidades para análise e resolução de problemas.
A ausência de disciplinas no ensino secundário que abordem a lógica também é um
fator determinante para os problemas encontrados. Mesmo aplicando a lógica no
dia a dia, grande parte dos estudantes possuem problemas ao aplicá-la de forma
satisfatória na construção de soluções algorítmicas sem os princípios básicos
que a norteiam.

% FIX ME REESCREVER com minhas palavras "PLÁGIO" de {miranda2004}
Por ser uma disciplina inicial, e uma disciplina com um grau de dificuldade
grande para os alunos que não conseguem se adaptar à forma de pensamento passo
a passo, muitas vezes o professor sente dificuldades, durante as aulas, de
avaliar qual é o problema encontrado pelos alunos. Normalmente as turmas de
algoritmos são turmas grandes, e a maior parte dos alunos não evidenciam de
forma verbal os problemas encontrados. Essas dúvidas somente se tornam claras
durante a aplicação de uma prova ou mesmo de exercícios válidos como nota
\cite{miranda2004}.

% FIX ME REESCREVER com minhas palavras "PLÁGIO" de {miranda2004}
Pode-se também citar, como problema, a questão da correção de provas e
exercícios que o professor aplica aos alunos. Como citado anteriormente,
normalmente as turmas da disciplina de Algoritmos são turmas grandes, primeiro
por ser uma disciplina ministrada no primeiro período - portanto, todos os
alunos que ingressam através do vestibular matriculam-se nela -, pelo alto
índice de reprovação. Ao corrigir provas de algoritmos, o professor deve levar
em consideração que cada aluno resolve problemas de formas diferentes,
portanto, um mesmo problema pode possuir diversos caminhos de resolução, e
avaliar todos os problemas um a um, estabelecendo um padrão para a correção.
Mas devido a grande quantidade de provas a serem corrigidas o professor,
involuntariamente, poderá alterar a forma de avaliação de um aluno para outro
\cite{miranda2004}.

% FIX ME REESCREVER com minhas palavras "PLÁGIO" de {miranda2004}
Apesar dos problemas citados anteriormente, a disciplina de algoritmos, baseada
em Portugol e fluxograma, ainda é ministrada na grande maioria dos cursos de
Computação, pois através dela o aluno consegue ter embasamento para aprender
diversas linguagens de programação. A disciplina de algoritmos em si ensina o
aluno a compreender um problema, definir sua resolução e aplicar esta resolução
no formato da pseudo linguagem. Na grande maioria das vezes, juntamente ao
ensino de Algoritmos é ensinado aos acadêmicos alguma linguagem de programação,
normalmente C ou Pascal. Com o algoritmo elaborado, torna-se mais fácil o
desenvolvimento do programa, pois a parte essencial, a parte lógica do problema,
já está resolvida. Portanto, o aluno deve somente transferir a resolução para a
linguagem de programação \cite{miranda2004}.

\section{O Ensino da Linguagem de Programação Portugol}

O ensino da lógica de programação geralmente é tratado nas primeiras fases dos
cursos de informática, onde os alunos iniciantes aprendem a desenvolver o
raciocínio lógico para então escrever algoritmos para solução de problemas. O
Portugol é uma pseudo linguagem algorítmica muito utilizada na descrição de
algoritmos, destaca-se pelo uso de comandos em português, o que facilita o
aprendizado da lógica de programação, habituando o iniciante com o formalismo de
programação.

% FIX ME REESCREVER com minhas palavras "PLÁGIO" de {118}
Apesar de todas essas vantagens, o Portugol apresenta o inconveniente dos
algoritmos não poderem ser executados no computador. Dessa forma, o iniciante
precisa imaginar a sua execução, o que não é uma tarefa tão fácil para quem está
começando.

% FIX ME REESCREVER com minhas palavras "PLÁGIO" de {118}
A lógica para programação consiste em aprender a pensar na mesma sequência em
que o computador executa as tarefas, aprende-se a imaginar como as ações serão
executadas partindo-se do estudo de um problema até chegar a construção de um
algoritmo (solução). Considere com exemplo o seguinte problema:

% FIX ME REESCREVER com minhas palavras "PLÁGIO" de {118}
``Expressar o resultado da soma de dois valores.''

% FIX ME REESCREVER com minhas palavras "PLÁGIO" de {118}
É comum, para uma pessoa pensar em algo assim:

% FIX ME REESCREVER com minhas palavras "PLÁGIO" de {118}
``Pegar os dois valores, somar e dar o resultado.''

% FIX ME REESCREVER com minhas palavras "PLÁGIO" de {118}
Após ter adquirido uma certa experiência, a mesma pessoa pode, de forma
automática, converter tal pensamento em instruções sem necessidade de especificar
detalhadamente os processos que estão implícitos nesta pequena rotina (ver tabela \ref{tab:action+operation}):

% FIX ME REESCREVER com minhas palavras "PLÁGIO" de {118}
\begin{table} [!htb]
  \caption{Ações e Operações}\label{tab:action+operation}
  \centering
  %\begin{tabular}{| m{2.4cm} | m{10.9cm}|}\hline
  \begin{tabular}{ m{2.4cm} | m{10.9cm} }\hline
  %\rowcolor{black!20}\Centering\bfseries Ações & \Centering\bfseries Operações \\ \hline
  \bfseries Ações & \bfseries Operações \\ \hline
  \MakeTextUppercase{Pegar} & Receber os dois valores numéricos e armazená-los \\ \hline
  \MakeTextUppercase{Somar} & Executar a instrução de soma e armazenar o resultado \\ \hline
  \MakeUppercase{Dar o \mbox{Resultado}} & Mostrar o resultado, armazená-lo para o uso
  posterior ou para ser visualizado em outra oportunidade \\
  \hline
  \end{tabular}
  \caption*{\footnotesize Fonte: Farrer (1989)}
\end{table}

% FIX ME REESCREVER com minhas palavras "PLÁGIO" de {118}
Quanto maior o domínio da lógica de programação, mais fácil será detalhar as
tarefas envolvidas na solução do problema proposto e mais eficiente será o
algoritmo criado, porém, para um iniciante construir um algoritmo que permita
a um computador executar a tarefa proposta não é tão simples.

% FIX ME REESCREVER com minhas palavras "PLÁGIO" de {118}
Segundo \cite[p. 17]{farreretal1989}: ``Algoritmo é a descrição de um conjunto de comandos que, obedecidos, resultam numa sucessão finita de ações''.
A tabela \ref{tab:action+operation}, mostra duas leituras de um algoritmos, uma bastante genérica (ações) e outro um pouco mais detalhada (operações), porém, apesar de apresentar um bom nível de detalhamento ainda não pode ser considerado um algoritmo para o computador. Em Portugol, o mesmo algoritmo pode ser escrito da seguinte forma:

%\DeclareRobustCommand{\Attribuition}{\text{\reflectbox{$\ding{212}$}}}

%\reflectbox{$\ding{212}$}
%\text{\reflectbox{$\ding{212}$}}

% FIX ME REESCREVER com minhas palavras "PLÁGIO" de {118}
\begin{lstlisting}[]  % Start your code-block

Algoritmo exemplo;
Var v1 ,v2 , v3 : inteiro;
Inicio
  leia (v1);
  leia (v2);
  v3 <- v1 + v2;
  escreva (v3);
Fim.
\end{lstlisting}

A representação dos algoritmos em Portugol, conhecido também como pseudocódigo,
é muito utilizada, segundo \cite[p.~6]{saliba1992}: ``Esta forma de
representação de algoritmo é rica em detalhes, como a definição dos tipos das
variáveis usadas no algoritmo e, por assemelhar-se bastante à forma em que os
programas são escritos, encontra muita aceitação.''

A tradução de um algoritmo escrito em Portugol para um programa de computador
numa linguagem de programação é muito fácil e clara, facilitando assim o ensino
e aprendizado da linguagem de programação. O Portugol é a forma mais utilizada
para escrever algoritmos por ser uma linguagem simples e permitir o
detalhamento dos algoritmos.

O ensino do Portugol, hoje, em muitos casos é feito de maneira manual
(utilizando folhas de papel), o que não estimula os alunos em aprender e
exercitar o desenvolvimento de algoritmos.



%% ----------------------------------------------------------

%% ELEMENTOS POS-TEXTUAIS

\postextual

\bibliographystyle{abntex2-alf}
\bibliography{refs}

\end{document}
