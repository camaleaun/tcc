
Praticar em \textit{software} à abstração em papel auxilia e muito no processo
de ensino e aprendizagem, em algoritmos além do pensamento lógico por etapas já
pode ser desenvolvido e descrito para o repasse a um computador.
``O Aprendizado de algoritmos não é uma tarefa muito fácil, só se consegue
através de muitos exercícios'' \cite{lopesgarcia2002}.

Para esse ensino é essencial conhecer dois conceitos principais lógica de
programação e algoritmos. Primeiramente é apresentado um paralelo de situações
do cotidiano com algoritmos, ao efetuar comparações de instruções passo a passo
realizadas diariamente, como fritar um ovo ou trocar um pneu de automóvel. A
seguir inicia-se a aplicação de linguagem estruturada em português para
descrição dos algoritmos. Em alguns casos, utiliza-se de linguagens em inglês,
mo por exemplo, Pascal. Há professores que preferem usar pseudo linguagem em
português somente ``em papel'', iniciando a codificação diretamente em linguagem
C. Tem-se inclusive, também ``em papel'', de analisar e interpretar os passos do
algoritmo para saber quais serão os valores em memória e a saída na tela.

Simultaneamente, com a prática escrita, sem software específico, pode-se simular
os mesmos algoritmos em interpretador ou compilador. Sendo em um interpretador
sua construção dividida em partes, cada uma com uma função específica.
Geralmente identifica-se: o analisador léxico, o analisador sintático, o
analisador semântico, e o resultado de saída. Caso seja um compilador há, ao
invés de resultado de saída, o gerador de código, que transforma o programa em
linguagem de máquina composta de 0s e 1s \cite{delamaro2004}.

Ao utilizar um programa em navegador via Internet, fazendo com que seja
desnecessário instalar e utilizar sempre o mesmo computador, objetiva-se que os
exercícios serão resolvidos com maior facilidade. Também é importante que as
mensagens de erro sejam claras, e em português já que torna mais acessível a
leitura para quem ainda não possui facilidade com a língua inglesa.

A criação do protótipo desse sistema é o ponto principal desse trabalho, pois os
objetivo é substituir \textit{softwares} que vêm sendo utilizados atualmente por
um prático, atual, acessível e com o código disponível para que possa ser
corrigido e melhorado.

Nesse protótipo busca-se eliminar algumas deficiências que existem em
ferramentas similares, pois essas podem fazer com que o seu uso seja desgastante
e desanimador. Além de propor a inclusão de funcionalidades que poderiam estar
disponíveis, as quais são encontradas por exemplo em Ambientes Integrados de
Programação, comumente utilizada em outras linguagens.


Inclusive há caso em que a ferramenta inicialmente utilizada não é capaz de
executar alguns algoritmos mais avançados. E isto leva a necessidade do uso
de outro software na mesma disciplina até com possivelmente uma nova
linguagem (por exemplo a linguagem de programação C ou C++), devido a está
necessidade alguns educadores optam por esta sendo a única ferramenta utilizada,
ignorando linguagem Portugol.
